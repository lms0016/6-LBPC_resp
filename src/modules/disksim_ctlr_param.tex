\noindent 
\begin{tabular}{|p{\lpmodwidth}|p{\lpnamewidth}|p{0.5in}|p{0.5in}|}
\cline{1-4}
\texttt{disksim\_ctlr} & \texttt{type} & int & required \\ 
\cline{1-4}
\multicolumn{4}{|p{6in}|}{
This specifies the type of controller. 1~indicates a simple
controller that acts as nothing more than a bridge between two buses,
passing everything straight through to the other side. 2~indicates a
very simple, driver-managed controller based roughly on the NCR
53C700. 3~indicates a more complex controller that decouples
lower-level storage component peculiarities from higher-level
components (e.g.,~device drivers). The complex controller queues and
schedules its outstanding requests and possibly contains a cache. As
indicated below, it requires several parameters in addition to those
needed by the simpler controllers.
}\\ 
\cline{1-4}
\multicolumn{4}{p{5in}}{}\\
\end{tabular}\\ 
\noindent 
\begin{tabular}{|p{\lpmodwidth}|p{\lpnamewidth}|p{0.5in}|p{0.5in}|}
\cline{1-4}
\texttt{disksim\_ctlr} & \texttt{Scale for delays} & float & required \\ 
\cline{1-4}
\multicolumn{4}{|p{6in}|}{
This specifies a multiplicative scaling factor for the various
processing delays incurred by the controller. Default overheads for
the 53C700-based controller and the more complex controller are
hard-coded into the ``read\_specs'' procedure of the controller module
(and are easily changed). For the simple pass-thru controller, the
scale factor represents the per-message propagation delay (because the
hard-coded value is 1.0). 0.0 results in no controller overheads or
delays. When the overheads/delays of the controller(s) cannot be
separated from those of the disk(s), as is usually the case for
single-point tracing of complete systems, the various disk
overhead/delay parameter values should be populated and this parameter
should be set to~0.0.
}\\ 
\cline{1-4}
\multicolumn{4}{p{5in}}{}\\
\end{tabular}\\ 
\noindent 
\begin{tabular}{|p{\lpmodwidth}|p{\lpnamewidth}|p{0.5in}|p{0.5in}|}
\cline{1-4}
\texttt{disksim\_ctlr} & \texttt{Bulk sector transfer time} & float & required \\ 
\cline{1-4}
\multicolumn{4}{|p{6in}|}{
This specifies the time (in milliseconds) necessary to transfer a
single 512-byte block to, from or through the controller.
Transferring one block over the bus takes the maximum of this time,
the block transfer time specified for the bus itself, and the block
transfer time specified for the component on the other end of the bus
transfer.
}\\ 
\cline{1-4}
\multicolumn{4}{p{5in}}{}\\
\end{tabular}\\ 
\noindent 
\begin{tabular}{|p{\lpmodwidth}|p{\lpnamewidth}|p{0.5in}|p{0.5in}|}
\cline{1-4}
\texttt{disksim\_ctlr} & \texttt{Maximum queue length} & int & required \\ 
\cline{1-4}
\multicolumn{4}{|p{6in}|}{
This specifies the maximum number of requests that can be concurrently
outstanding at the controller. The device driver discovers this value
during initialization and respects it during operation. For the
simple types of controllers (see above parameter description), 0 is
assumed.
}\\ 
\cline{1-4}
\multicolumn{4}{p{5in}}{}\\
\end{tabular}\\ 
\noindent 
\begin{tabular}{|p{\lpmodwidth}|p{\lpnamewidth}|p{0.5in}|p{0.5in}|}
\cline{1-4}
\texttt{disksim\_ctlr} & \texttt{Print stats} & int & required \\ 
\cline{1-4}
\multicolumn{4}{|p{6in}|}{
This specifies whether or not statistics will be reported for the
controller. It is meaningless for the simple types of controllers
(see above parameter description), as no statistics are collected.
}\\ 
\cline{1-4}
\multicolumn{4}{p{5in}}{}\\
\end{tabular}\\ 
\noindent 
\begin{tabular}{|p{\lpmodwidth}|p{\lpnamewidth}|p{0.5in}|p{0.5in}|}
\cline{1-4}
\texttt{disksim\_ctlr} & \texttt{Scheduler} & block & optional \\ 
\cline{1-4}
\multicolumn{4}{|p{6in}|}{
This is an ioqueue; see section \ref{param.queue}
}\\ 
\cline{1-4}
\multicolumn{4}{p{5in}}{}\\
\end{tabular}\\ 
\noindent 
\begin{tabular}{|p{\lpmodwidth}|p{\lpnamewidth}|p{0.5in}|p{0.5in}|}
\cline{1-4}
\texttt{disksim\_ctlr} & \texttt{Cache} & block & optional \\ 
\cline{1-4}
\multicolumn{4}{|p{6in}|}{
A block cache; see section \ref{param.cache}
}\\ 
\cline{1-4}
\multicolumn{4}{p{5in}}{}\\
\end{tabular}\\ 
\noindent 
\begin{tabular}{|p{\lpmodwidth}|p{\lpnamewidth}|p{0.5in}|p{0.5in}|}
\cline{1-4}
\texttt{disksim\_ctlr} & \texttt{Max per-disk pending count} & int & optional \\ 
\cline{1-4}
\multicolumn{4}{|p{6in}|}{
This specifies the maximum number of requests that the controller can
have outstanding to each attached disk (i.e.,~the maximum number of
requests that can be dispatched to a single disk). This parameter
only affects the interaction of the controller with its attachments;
it is not visible to the device driver.
}\\ 
\cline{1-4}
\multicolumn{4}{p{5in}}{}\\
\end{tabular}\\ 

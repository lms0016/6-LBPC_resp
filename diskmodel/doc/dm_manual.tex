\documentclass{article}
\setlength{\textwidth}{6in}
\setlength{\oddsidemargin}{0in}
\setlength{\evensidemargin}{0in}

\newcommand{\lpnamewidth}{2 in}
\newcommand{\lpmodwidth}{1.5 in}

\begin{document}
\title{Diskmodel}
\maketitle

\section{Overview}

\subsection{Introduction}
Diskmodel is a library implementing mechanical and layout models of
modern magnetic disk drives.  Diskmodel models two major aspects of disk
operation.  The layout module models logical-to-physical mapping of
blocks, defect management and also computes angular offsets of blocks.
The mechanical model handles seek times, rotational latency and
various other aspects of disk mechanics.

The implementations of these modules in the current version of
Diskmodel are derived from DiskSim 2.0 \cite{Ganger99}.  Disksim 3.0
uses Diskmodel natively. Diskmodel has also been used in a device
driver implementation of a shortest positioning time first disk
request scheduler.


\subsection{Types and Units}

All math in diskmodel is performed using integer arithmetic.  Angles
identified as points on a circle divided into discrete units.  Time is
represented as multiples of some very small time base.  Diskmodel
exports the types \texttt{dm\_time\_t} and \texttt{dm\_angle\_t} to
represent these quantities.  Diskmodel exports functions
\texttt{dm\_time\_itod}, \texttt{dm\_time\_dtoi} (likewise for angles)
for converting between doubles and the native format.  The time
function converts to and from milliseconds; the angle function
converts to and from a fraction of a circle.  \texttt{dm\_time\_t} and
\texttt{dm\_angle\_t} should be regarded as opaque and may change over
time.  Diskmodel is sector-size agnostic in that it assumes that
sectors are some fixed size but does not make any assumption about
what that size is.

\subsubsection{Three Zero Angles}

When considering the angular offset of a sector on a track, there are
at least three plausible candidates for a ``zero'' angle.  The first
is ``absolute'' zero which is the same on every track on the disk.
For various reasons, this zero may not coincide with a sector
boundary on a track.  This motivates the second 0 which we will refer
to as $0_t$ (t for ``track'') which is the angular offset of the first
sector boundary past 0 on a track.  Because of skews and defects, the
lowest lbn on the track may not lie at $0_t$.  We call the angle of
the lowest sector on the track $0_l$ (l for ``logical'' or ``lbn'').

\subsubsection{Two Zero Sectors}

Similarly, when numbering the sectors on a track, it is reasonable to
call either the sector at $0_t$ or the one at $0_l$ ``sector 0.''
$0_t$ corresponds to directly to the physical location of sectors on a
track whereas $0_l$ corresponds to logical layout.  Diskmodel works in
both systems and the following function descriptions identify which
numbering a given function uses.

\subsubsection{Example}

Consider a disk with 100 sectors per track, 2 heads, a head switch
skew of 10 sectors and a cylinder switch skew of 20 sectors. $(x,y,z)$
denotes cylinder $x$, head $y$ and sector $z$.\\

\begin{tabular}{l|l|l}
LBN & $0_l$ PBN & $0_t$ PBN \\
\cline{1-3}
0 & (0,0,0) & (0,0,0) \\
\multicolumn{3}{c}{$\vdots$} \\
99 & (0,0,99) & (0,0,99) \\
100 &  (0,1,0) & (0,1,10) \\
101 &  (0,1,1) & (0,1,11) \\
\multicolumn{3}{c}{$\vdots$} \\
189 & (0,1,89) & (0,1,99) \\
190 & (0,1,90) & (0,1,0) \\
191 & (0,1,91) & (0,1,1) \\
199 & (0,1,99) & (0,1,9) \\
\end{tabular}\\

Note that a sector is $3.6$ degrees wide.\\

\begin{tabular}{l|l|l}
Cylinder & Head & $0_l$ angle \\
\cline{1-3}
0 & 0 & 0 degrees \\
0 & 1 & 36 degrees \\
1 & 0 & 72 degrees \\
1 & 1 & 108 degrees \\
2 & 0 & 180 degrees \\
\end{tabular}


\subsection{API}

This section describes the data structures and functions that comprise
the Diskmodel API.

The \texttt{dm\_disk\_if} struct is the ``top-level'' handle for a
disk in diskmodel.  It contains a few disk-wide parameters -- number
of heads/surfaces, cylinders and number of logical blocks exported by
device -- along with pointers to the mechanics and layout interfaces.


\subsubsection{Disk-wide Parameters}

The top-level of a disk model is the \texttt{dm\_disk\_if} struct:

\begin{verbatim}
struct dm_disk_if {
  int dm_cyls;            // number of cylinders
  int dm_surfaces;        // number of media surfaces used for data
  int dm_sectors;         // LBNs or total physical sectors (??)

  struct dm_layout_if   *layout;
  struct dm_mech_if     *mech;
};
\end{verbatim}

All fields of diskmodel API structures are read-only; the behavior of
diskmodel after any of them is modified is undefined.  \texttt{layout}
and \texttt{mech} are pointers to the layout and mechanical module
interfaces, respectively.  Each is a structure containing a number of
pointers to functions which constitute the actual implementation.  In
the following presentation, we write the functions as declarations
rather than as types of function pointers for readability.  Many of the
methods take one or more result parameters; i.e. pointers whose
addresses will be filled in with some result.  Unless otherwise
specified, passing \texttt{NULL} for result parameters is allowed and
the result will not be filled in.


\subsubsection{Layout}

The layout interface uses the following auxiliary type:

\texttt{dm\_ptol\_result\_t} appears in situations where a client code
provides a pbn which may not exist on disk as-described e.g. due to
defects.  It contains the following values:

\begin{verbatim}
  DM_SLIPPED
  DM_REMAPPED
  DM_OK
  DM_NX
\end{verbatim}

\texttt{DM\_SLIPPED} indicates that the pbn is a slipped defect.
\texttt{DM\_REMAPPED} indicates that the pbn is a remapped defect.
\texttt{DM\_OK} indicates that the pbn exists on disk as-is.  \texttt{DM\_NX}
indicates that there is no sector on the device corresponding to the
given pbn.  When interpreted as integers, these values are all less
than zero so they can be unambiguously intermixed with nonnegative
integers e.g. lbns.



The layout module exports the following methods:

\begin{verbatim}
dm_ptol_result_t dm_translate_ltop(struct dm_disk_if *, 
                                   int lbn, 
                                   dm_layout_maptype,
                                   struct dm_pbn *result,
                                   int *remapsector);
\end{verbatim}

Translate a logical block number (lbn) to a physical block number
(pbn).  \texttt{remapsector} is a result parameter which will be set
to a non-zero value if the lbn was remapped.

The sector number in the result is relative to the $0_l$ zero sector.




\begin{verbatim}
dm_ptol_result_t dm_translate_ltop_0t(struct dm_disk_if *, 
                                     int lbn, 
                                     dm_layout_maptype,
                                     struct dm_pbn *result,
                                     int *remapsector);
\end{verbatim}


Same as \texttt{dm\_translate\_ltop} except that the sector in 
result is relative to the $0_t$ sector.



\begin{verbatim}
dm_ptol_result_t dm_translate_ptol(struct dm_disk_if *, 
                                   struct dm_pbn *p,
                                   int *remapsector);
\end{verbatim}

Translate a pbn to an lbn.  \texttt{remapsector} is a result parameter
which will be set to a non-zero value if the pbn is defective and
remapped.

The sector number in the operand is relative to the $0_l$ zero sector.

\begin{verbatim}
dm_ptol_result_t dm_translate_ptol_0t(struct dm_disk_if *, 
                                      struct dm_pbn *p,
                                      int *remapsector);
\end{verbatim}

Same as \texttt{dm\_translate\_ptol} except that the sector in the result
is relative to the $0_t$ sector.



\begin{verbatim}
int dm_get_sectors_lbn(struct dm_disk_if *d,
                       int lbn);
\end{verbatim}

Returns the number of sectors on the track containing the given lbn.


\begin{verbatim}
int dm_get_sectors_pbn(struct dm_disk_if *d,
                       struct dm_pbn *);
\end{verbatim}

Returns the number of physical sectors on the track containing the
given pbn.  This may not be the same as the number of lbns mapped on
this track.  If the cylinder is unmapped, the return value will be the
number of sectors per track for the nearest (lower) zone.


\begin{verbatim}
void dm_get_track_boundaries(struct dm_disk_if *d,
                             struct dm_pbn *,
                             int *first_lbn,
                             int *last_lbn,
                             int *remapsector);
\end{verbatim}

Computes lbn boundaries for the track containing the given pbn.
\texttt{first\_lbn} is a result parameter which returns the first lbn
on the track containing the given pbn; similarly, \texttt{last\_lbn}
returns the last lbn on the given track.  \texttt{remapsector} returns
a non-zero value if the first or last block on the track are remapped.
Note that \texttt{last\_lbn} - \texttt{first\_lbn} + 1 may be greater
than the number of LBNs mapped on the track e.g. due to remapped defects.


\begin{verbatim}
dm_ptol_result_t dm_seek_distance(struct dm_disk_if *,
                                  int start_lbn,
                                  int dest_lbn);
\end{verbatim}

Computes the seek distance in cylinders that would be incurred for
given request.  Returns a \texttt{dm\_ptol\_result\_t} since one or both of the
LBNs may be slipped or remapped.


\begin{verbatim}
dm_angle_t dm_pbn_skew(struct dm_disk_if *,
                       struct dm_pbn *);
\end{verbatim}

This computes the starting offset of a pbn relative to 0.  The operand
is a pbn relative to $0_l$; the result is an angle relative to $0$.
This accounts for all skews, slips, etc.

\begin{verbatim}
dm_angle_t dm_get_track_zerol(struct dm_disk_if *, 
                              struct dm_mech_state *);
\end{verbatim}

The return value is $0_l$ for the track identified by the second
argument.  This is equivalent to calling \texttt{dm\_pbn\_skew}
for sector 0 on the same track.


\begin{verbatim}
dm_ptol_result_t dm_convert_atop(struct dm_disk_if *,
                                 struct dm_mech_state *,
                                 struct dm_pbn *);
\end{verbatim}

Finds the pbn of the sector whose leading edge is less than or equal
to the given angle.  Returns a \texttt{ptol\_result\_t} since the
provided angle could be in slipped space, etc.  Both the angle in the
second operand and the sector number in the result pbn are relative to
$0_l$.



\begin{verbatim}
dm_angle_t dm_get_sector_width(struct dm_disk_if *,
                               struct dm_pbn *track,
                               int num);
\end{verbatim}

Returns the angular width of an extent of num sectors on the given track.
Returns 0 if \texttt{num} is greater than the number of sectors on the
track.

\begin{verbatim}
dm_angle_t dm_lbn_offset(struct dm_disk_if *, int lbn1, int lbn2);
\end{verbatim}

Computes the angular distance/offset between two logical blocks.


\begin{verbatim}
int dm_marshalled_len(struct dm_disk_if *);
\end{verbatim}

Returns the size of the structure in bytes when marshalled.


\begin{verbatim}
void *dm_marshall(struct dm_disk_if *, char *);
\end{verbatim}

Marshall this layout struct into the provided buffer.  The return value is
the first address in the buffer not written.


\subsubsection{Mechanics}

The following diagram shows the breakdown of a zero-latency access in
our model, and the corresponding definitions of seek time, positioning
time and access time.

\begin{verbatim}
+-------------------------+------------+----------+---------+----------+
| seek                    | initial    |          | add.    |          |
| headswitch              | rotational | xfertime | rot.    | xfertime |
|            extra settle | latency    |          | latency |          |
+-------------------------+------------+----------+---------+----------+

|---------seektime--------|
|-----------positioning-time-----------|
|------------------------------access-time-----------------------------|
\end{verbatim}

\begin{verbatim}
dm_time_t dm_seek_time(struct dm_disk_if *, 
                       struct dm_mech_state *start_track,
                       struct dm_mech_state *end_track,
                       int read);
\end{verbatim}

Computes the amount of time to seek from the first track to the second
track, possibly including a head switch and additional write settling
time.  This is only track-to-track so the angles in the parameters are
ignored.  \texttt{read} should be nonzero if the access on the destination
track is a read and zero if it is a write; extra write-settle time is
included in the result for writes.

\begin{verbatim}
int dm_access_block(struct dm_disk_if *,
                    struct dm_mech_state *initial,
                    int start,
                    int len,
                    int immed);
\end{verbatim}

From the given inital condition and access, it will return the first
block on the track to be read.  The access is for \texttt{len} sectors
starting at physical sector \texttt{start} on the same track as
\texttt{initial}.  \texttt{immed} indicates if this is an
``immediate'' or ``zero-latency'' access; if \texttt{immed} is zero,
the result will always be the same as \texttt{start}.


\begin{verbatim}
dm_time_t dm_latency(struct dm_disk_if *, 
                     struct dm_mech_state *initial,
                     int start,                    
                     int len,                      
                     int immed,                    
                     dm_time_t *addtolatency); 
\end{verbatim}

This computes the rotational latency incurred from accessing up to
\texttt{len} blocks from the track starting from angle
\texttt{initial} and sector \texttt{start}.  This will access to the
end of the track but not wrap around; e.g. for a sequential access that
starts on the given track and switches to another, after reaching the
end of the first.  The return value is the initial rotational latency;
i.e. how long before the media transfer for the first block to be read
starts.  \texttt{addtolatency} is a result parameter returning
additional rotational latency as defined in the figure above.  Note
that for non-zero-latency accesses, addtolatency will always be zero.
Also note that for zero latency accesses, the latency is the amount of
time before the media transfer begins for the first sector i.e. the
same sector that would be returned by \texttt{dm\_access\_block()}.


\texttt{dm\_pos\_time} and \texttt{dm\_acctime} optionally return
broken-down components of the result via the following struct:

\begin{verbatim}
struct dm_mech_acctimes {
   dm_time_t seektime;
   dm_time_t initial_latency;
   dm_time_t initial_xfer;
   dm_time_t addl_latency;
   dm_time_t addl_xfer;
};
\end{verbatim}

For a zero-latency access, the last two fields will always be zero.
\texttt{dm\_pos\_time} only fills in the first two fields;
\texttt{dm\_acctime} fills in all 5.

\begin{verbatim}
dm_time_t dm_pos_time(struct dm_disk_if *,
                      struct dm_mech_state *initial,
                      struct dm_pbn *start,
                      int len,
                      int rw,
                      int immed);
\end{verbatim}

Compute the amount of time before the media transfer for \texttt{len}
sectors starting at \texttt{start} begins starting with the disk
mechanics in state \texttt{initial}.  0 for \texttt{rw} indicates a
write, any other value indicates a read.  A non-zero value for
\texttt{immed} indicates a ``zero-latency'' access.  Positioning time
is the same as seek time (including head-switch time and any extra
write-settle time) plus initial rotational latency.

\texttt{len} must be at least 1.

\begin{verbatim}
dm_time_t dm_acctime(struct dm_disk_if *, 
                     struct dm_mech_state *initial_state,
                     struct dm_pbn *start,
                     int len,
                     int rw,
                     int immed,
                     struct dm_mech_state *result_state);
\end{verbatim}

Estimate how long it will take to access \texttt{len} sectors starting
with pbn \texttt{start} with the disk initially in state
\texttt{initial}. 0 for \texttt{rw} indicates a write; any other value
indicates a read.  A non-zero value for \texttt{immed} indicates a
``zero-latency'' access.  \texttt{result\_state} is a result parameter
which returns the mechanical state of the disk when the access
completes.

\texttt{len} must be at least 1.

Access time consists of positioning time (above), transfer time and
any additional rotational latency not included in the positioning
time, e.g. in the middle of a zero-latency access transfer. 

\texttt{dm\_acctime} ignores defects so it yields a smaller-than-correct 
result when computing access times on tracks with defective sectors.
This is deliberate as the handling of defects is a high-level
controller function which varies widely.

\begin{verbatim}
dm_time_t dm_rottime(struct dm_disk_if *,
                     dm_angle_t begin,
                     dm_angle_t end);
\end{verbatim}

Compute how long it will take the disk to rotate from the angle in the
first position to that in the second position.
  

\begin{verbatim}
dm_time_t dm_xfertime(struct dm_disk_if *d,
                      struct dm_mech_state *,
                      int len);
\end{verbatim}

Computes the amount of time to transfer len sectors to or from the
track designated by the second argument.  This is computed in terms of
\texttt{dm\_get\_sector\_width()} and \texttt{dm\_rottime()} in the
obvious way.


\begin{verbatim}
dm_time_t dm_headswitch_time(struct dm_disk_if *, 
                             int h1, 
                             int h2);
\end{verbatim}

Returns the amount of time to swith from using the first head to the
second. 


\begin{verbatim}
dm_angle_t dm_rotate(struct dm_disk_if *, 
                     dm_time_t *time);
\end{verbatim}

Returns the angle of the media after \texttt{time} has elapsed
assuming the media started at angle 0.


\begin{verbatim}
dm_time_t dm_period(struct dm_disk_if *);
\end{verbatim}

Returns the rotational period of the media.


\begin{verbatim}
int dm_marshalled_len(struct dm_disk_if *);
\end{verbatim}

Returns the marshalled size of the structure.


\begin{verbatim}
void *dm_marshall(struct dm_disk_if *, char *);
\end{verbatim}

Marshalls the structure into the given buffer.  The return value is
the first address in the buffer not written.


\subsection{Model Configuration}
Diskmodel uses libparam to input the following blocks of parameter data:

\begin{verbatim}
dm_disk
dm_layout_g1
dm_layout_g1_zone
dm_mech_g1
dm_layout_g2
dm_layout_g2_zone
dm_layout_g4
\end{verbatim}

\subsubsection{dm\_disk}

The outer \texttt{dm\_disk} block contains the top-level parameters
which are used to fill in the \texttt{dm\_disk\_if} structure.  The
only valid value for ``Layout Model'' is a \texttt{dm\_layout\_g1}
block and for ``Mechanical Model,'' a \texttt{dm\_mech\_g1} block.\\

\noindent 
\begin{tabular}{|p{\lpmodwidth}|p{\lpnamewidth}|p{0.5in}|p{0.5in}|}
\cline{1-4}
\texttt{dm\_disk} & \texttt{Block count} & int & required \\ 
\cline{1-4}
\multicolumn{4}{|p{6in}|}{
This specifies the number of data blocks. This capacity is exported by the
disk (e.g.,~to a disk array controller). It is not used directly
during simulation, but is compared to a similar value computed from
other disk parameters. A warning is reported if the values differ.
}\\ 
\cline{1-4}
\multicolumn{4}{p{5in}}{}\\
\end{tabular}\\ 
\noindent 
\begin{tabular}{|p{\lpmodwidth}|p{\lpnamewidth}|p{0.5in}|p{0.5in}|}
\cline{1-4}
\texttt{dm\_disk} & \texttt{Number of data surfaces} & int & required \\ 
\cline{1-4}
\multicolumn{4}{|p{6in}|}{
This specifies the number of magnetic media surfaces (not platters!) on
which data are recorded. Dedicated servo surfaces should not be
counted for this parameter.
}\\ 
\cline{1-4}
\multicolumn{4}{p{5in}}{}\\
\end{tabular}\\ 
\noindent 
\begin{tabular}{|p{\lpmodwidth}|p{\lpnamewidth}|p{0.5in}|p{0.5in}|}
\cline{1-4}
\texttt{dm\_disk} & \texttt{Number of cylinders} & int & required \\ 
\cline{1-4}
\multicolumn{4}{|p{6in}|}{
This specifies the number of physical cylinders. All cylinders that impact
the logical to physical mappings should be included.
}\\ 
\cline{1-4}
\multicolumn{4}{p{5in}}{}\\
\end{tabular}\\ 
\noindent 
\begin{tabular}{|p{\lpmodwidth}|p{\lpnamewidth}|p{0.5in}|p{0.5in}|}
\cline{1-4}
\texttt{dm\_disk} & \texttt{Mechanical Model} & block & optional \\ 
\cline{1-4}
\multicolumn{4}{|p{6in}|}{
This block defines the disk's mechanical model. Currently,
the only available implementation is \texttt{dm\_mech\_g1}.
}\\ 
\cline{1-4}
\multicolumn{4}{p{5in}}{}\\
\end{tabular}\\ 
\noindent 
\begin{tabular}{|p{\lpmodwidth}|p{\lpnamewidth}|p{0.5in}|p{0.5in}|}
\cline{1-4}
\texttt{dm\_disk} & \texttt{Layout Model} & block & required \\ 
\cline{1-4}
\multicolumn{4}{|p{6in}|}{
This block defines the disk's layout model.
}\\ 
\cline{1-4}
\multicolumn{4}{p{5in}}{}\\
\end{tabular}\\ 


\subsubsection{G1 Layout}
% \subsubsection{dm\_layout\_g1}

The \texttt{dm\_layout\_g1} block provides parameters for a first
generation (g1) layout model.\\

\noindent 
\begin{tabular}{|p{\lpmodwidth}|p{\lpnamewidth}|p{0.5in}|p{0.5in}|}
\cline{1-4}
\texttt{dm\_layout\_g1} & \texttt{LBN-to-PBN mapping scheme} & int & required \\ 
\cline{1-4}
\multicolumn{4}{|p{6in}|}{
This specifies the type of LBN-to-PBN mapping used by the disk.
0 indicates that the conventional mapping scheme is used:
LBNs advance along the 0th track of the 0th cylinder, then along the
1st track of the 0th cylinder, thru the end of the 0th cylinder, then
to the 0th track of the 1st cylinder, and so forth.
1 indicates that the conventional mapping scheme is modified slightly,
such that cylinder switches do not involve head switches. Thus, after
LBNs are assigned to the last track of the 0th cylinder, they are
assigned to the last track of the 1st cylinder, the next-to-last track
of the 1st cylinder, thru the 0th track of the 1st cylinder. LBNs are
then assigned to the 0th track of the 2nd cylinder, and so on
(``first cylinder is normal'').
2 is like 1 except that the serpentine pattern does not reset at the
beginning of each zone; rather, even cylinders are always ascending and
odd cylinders are always descending.
}\\ 
\cline{1-4}
\multicolumn{4}{p{5in}}{}\\
\end{tabular}\\ 
\noindent 
\begin{tabular}{|p{\lpmodwidth}|p{\lpnamewidth}|p{0.5in}|p{0.5in}|}
\cline{1-4}
\texttt{dm\_layout\_g1} & \texttt{Sparing scheme used} & int & required \\ 
\cline{1-4}
\multicolumn{4}{|p{6in}|}{
This specifies the type of sparing used by the disk. Later parameters determine
where spare space is allocated.
0~indicates that no spare sectors are allocated.
1~indicates that entire tracks of spare sectors are allocated at the ``end''
of some or all zones (sets of cylinders).
2~indicates that spare sectors are allocated at the ``end'' of each cylinder.
3~indicates that spare sectors are allocated at the ``end'' of each track.
4~indicates that spare sectors are allocated at the ``end'' of each cylinder
and that slipped sectors do not utilize these spares (more spares are located
at the ``end'' of the disk).
5~indicates that spare sectors are allocated at the ``front'' of each cylinder.
6~indicates that spare sectors are allocated at the ``front'' of each cylinder
and that slipped sectors do not utilize these spares (more spares are located
at the ``end'' of the disk).
7~indicates that spare sectors are allocated at the ``end'' of the disk.
8~indicates that spare sectors are allocated at the ``end'' of each range
of cylinders.
9~indicates that spare sectors are allocated at the ``end'' of each zone.
10~indicates that spare sectors are allocated at the ``end'' of each zone
and that slipped sectors do not use these spares (more spares are located
at the ``end'' of the disk).
}\\ 
\cline{1-4}
\multicolumn{4}{p{5in}}{}\\
\end{tabular}\\ 
\noindent 
\begin{tabular}{|p{\lpmodwidth}|p{\lpnamewidth}|p{0.5in}|p{0.5in}|}
\cline{1-4}
\texttt{dm\_layout\_g1} & \texttt{Rangesize for sparing} & int & required \\ 
\cline{1-4}
\multicolumn{4}{|p{6in}|}{
This specifies the range (e.g., of cylinders) over which spares are
allocated and maintained. Currently, this value is relevant only for
disks that use ``sectors per cylinder range'' sparing schemes.
}\\ 
\cline{1-4}
\multicolumn{4}{p{5in}}{}\\
\end{tabular}\\ 
\noindent 
\begin{tabular}{|p{\lpmodwidth}|p{\lpnamewidth}|p{0.5in}|p{0.5in}|}
\cline{1-4}
\texttt{dm\_layout\_g1} & \texttt{Skew units} & string & optional \\ 
\cline{1-4}
\multicolumn{4}{|p{6in}|}{
This sets the units with which units are input: \texttt{revolutions} or
\texttt{sectors}. The ``disk-wide'' value set here may be overridden
per-zone. The default unit is \texttt{sectors}.
}\\ 
\cline{1-4}
\multicolumn{4}{p{5in}}{}\\
\end{tabular}\\ 
\noindent 
\begin{tabular}{|p{\lpmodwidth}|p{\lpnamewidth}|p{0.5in}|p{0.5in}|}
\cline{1-4}
\texttt{dm\_layout\_g1} & \texttt{Zones} & list & required \\ 
\cline{1-4}
\multicolumn{4}{|p{6in}|}{
This is a list of zone block values describing the zones/bands of the disk.
}\\ 
\cline{1-4}
\multicolumn{4}{p{5in}}{}\\
\end{tabular}\\ 


The \texttt{Zones} parameter is a list of zone blocks each of which
contains the following fields:\\

\noindent 
\begin{tabular}{|p{\lpmodwidth}|p{\lpnamewidth}|p{0.5in}|p{0.5in}|}
\cline{1-4}
\texttt{dm\_layout\_g1\_zone} & \texttt{First cylinder number} & int & required \\ 
\cline{1-4}
\multicolumn{4}{|p{6in}|}{
This specifies the first physical cylinder in the zone.
}\\ 
\cline{1-4}
\multicolumn{4}{p{5in}}{}\\
\end{tabular}\\ 
\noindent 
\begin{tabular}{|p{\lpmodwidth}|p{\lpnamewidth}|p{0.5in}|p{0.5in}|}
\cline{1-4}
\texttt{dm\_layout\_g1\_zone} & \texttt{Last cylinder number} & int & required \\ 
\cline{1-4}
\multicolumn{4}{|p{6in}|}{
This specifies the last physical cylinder in the zone.
}\\ 
\cline{1-4}
\multicolumn{4}{p{5in}}{}\\
\end{tabular}\\ 
\noindent 
\begin{tabular}{|p{\lpmodwidth}|p{\lpnamewidth}|p{0.5in}|p{0.5in}|}
\cline{1-4}
\texttt{dm\_layout\_g1\_zone} & \texttt{Blocks per track} & int & required \\ 
\cline{1-4}
\multicolumn{4}{|p{6in}|}{
This specifies the number of sectors (independent of logical-to-physical
mappings) on each physical track in the zone.
}\\ 
\cline{1-4}
\multicolumn{4}{p{5in}}{}\\
\end{tabular}\\ 
\noindent 
\begin{tabular}{|p{\lpmodwidth}|p{\lpnamewidth}|p{0.5in}|p{0.5in}|}
\cline{1-4}
\texttt{dm\_layout\_g1\_zone} & \texttt{Offset of first block} & float & required \\ 
\cline{1-4}
\multicolumn{4}{|p{6in}|}{
This specifies the physical offset of the first logical sector in the
zone. Physical sector 0 of every track is assumed to begin at the
same angle of rotation. This may be in either sectors or revolutions
according to the ``Skew units'' parameter.
}\\ 
\cline{1-4}
\multicolumn{4}{p{5in}}{}\\
\end{tabular}\\ 
\noindent 
\begin{tabular}{|p{\lpmodwidth}|p{\lpnamewidth}|p{0.5in}|p{0.5in}|}
\cline{1-4}
\texttt{dm\_layout\_g1\_zone} & \texttt{Skew units} & string & optional \\ 
\cline{1-4}
\multicolumn{4}{|p{6in}|}{
Default is \texttt{sectors}. This value overrides any set in the
surrounding layout block.
}\\ 
\cline{1-4}
\multicolumn{4}{p{5in}}{}\\
\end{tabular}\\ 
\noindent 
\begin{tabular}{|p{\lpmodwidth}|p{\lpnamewidth}|p{0.5in}|p{0.5in}|}
\cline{1-4}
\texttt{dm\_layout\_g1\_zone} & \texttt{Empty space at zone front} & int & required \\ 
\cline{1-4}
\multicolumn{4}{|p{6in}|}{
This specifies the size of the ``management area'' allocated at the
beginning of the zone for internal data structures. This area can not
be accessed during normal activity and is not part of the disk's
logical-to-physical mapping.
}\\ 
\cline{1-4}
\multicolumn{4}{p{5in}}{}\\
\end{tabular}\\ 
\noindent 
\begin{tabular}{|p{\lpmodwidth}|p{\lpnamewidth}|p{0.5in}|p{0.5in}|}
\cline{1-4}
\texttt{dm\_layout\_g1\_zone} & \texttt{Skew for track switch} & float & optional \\ 
\cline{1-4}
\multicolumn{4}{|p{6in}|}{
This specifies the number of physical sectors that are skipped when
assigning logical block numbers to physical sectors at a track
crossing point. Track skew is computed by the manufacturer to
optimize sequential access. This may be in either sectors or
revolutions according to the ``Skew units'' parameter.
}\\ 
\cline{1-4}
\multicolumn{4}{p{5in}}{}\\
\end{tabular}\\ 
\noindent 
\begin{tabular}{|p{\lpmodwidth}|p{\lpnamewidth}|p{0.5in}|p{0.5in}|}
\cline{1-4}
\texttt{dm\_layout\_g1\_zone} & \texttt{Skew for cylinder switch} & float & optional \\ 
\cline{1-4}
\multicolumn{4}{|p{6in}|}{
This specifies the number of physical sectors that are skipped when
assigning logical block numbers to physical sectors at a cylinder
crossing point. Cylinder skew is computed by the manufacturer to
optimize sequential access. This may be in either sectors or
revolutions according to the ``Skew units'' parameter.
}\\ 
\cline{1-4}
\multicolumn{4}{p{5in}}{}\\
\end{tabular}\\ 
\noindent 
\begin{tabular}{|p{\lpmodwidth}|p{\lpnamewidth}|p{0.5in}|p{0.5in}|}
\cline{1-4}
\texttt{dm\_layout\_g1\_zone} & \texttt{Number of spares} & int & required \\ 
\cline{1-4}
\multicolumn{4}{|p{6in}|}{
This specifies the number of spare storage locations -- sectors or tracks,
depending on the sparing scheme chosen -- allocated per region of
coverage which may be a track, cylinder, or zone, depending on the
sparing scheme. For example, if the sparing scheme is 1, indicating
that spare tracks are allocated at the end of the zone, the value of
this parameter indicates how many spare tracks have been allocated for
this zone.
}\\ 
\cline{1-4}
\multicolumn{4}{p{5in}}{}\\
\end{tabular}\\ 
\noindent 
\begin{tabular}{|p{\lpmodwidth}|p{\lpnamewidth}|p{0.5in}|p{0.5in}|}
\cline{1-4}
\texttt{dm\_layout\_g1\_zone} & \texttt{slips} & list & required \\ 
\cline{1-4}
\multicolumn{4}{|p{6in}|}{
This is a list of lbns for previously detected defective media
locations -- sectors or tracks, depending upon the sparing scheme
chosen -- that were skipped-over or ``slipped'' when the
logical-to-physical mapping was last created. Each integer in the
list indicates the slipped (defective) location.
}\\ 
\cline{1-4}
\multicolumn{4}{p{5in}}{}\\
\end{tabular}\\ 
\noindent 
\begin{tabular}{|p{\lpmodwidth}|p{\lpnamewidth}|p{0.5in}|p{0.5in}|}
\cline{1-4}
\texttt{dm\_layout\_g1\_zone} & \texttt{defects} & list & required \\ 
\cline{1-4}
\multicolumn{4}{|p{6in}|}{
This list describes previously detected defective media
locations -- sectors or tracks, depending upon the sparing scheme
chosen -- that have been remapped to alternate physical locations.
The elements of the list are interpreted as pairs wherein the first
number is the original (defective) location and the second number
indicates the replacement location. Note that these locations
will both be either a physical sector number or a physical track
number, depending on the sparing scheme chosen.
}\\ 
\cline{1-4}
\multicolumn{4}{p{5in}}{}\\
\end{tabular}\\ 


\subsubsection{G1 Mechanics}
%\subsubsection{dm\_mech\_g1}

The \texttt{dm\_mech\_g1} block provides parameters for a first
generation (g1) mechanical model.\\

\noindent 
\begin{tabular}{|p{\lpmodwidth}|p{\lpnamewidth}|p{0.5in}|p{0.5in}|}
\cline{1-4}
\texttt{dm\_mech\_g1} & \texttt{Access time type} & string & required \\ 
\cline{1-4}
\multicolumn{4}{|p{6in}|}{
This specifies the method for computing mechanical delays. Legal values
are \texttt{constant} which indicates a fixed per-request access time
(i.e.,~actual mechanical activity is not modeled),
\texttt{averageRotation} which indicates that seek activity should be
modeled but rotational latency is assumed to be equal to one half of
a rotation (the statistical mean for random disk access) and
\texttt{trackSwitchPlusRotation} which indicates that both seek and
rotational activity should be modeled.
}\\ 
\cline{1-4}
\multicolumn{4}{p{5in}}{}\\
\end{tabular}\\ 
\noindent 
\begin{tabular}{|p{\lpmodwidth}|p{\lpnamewidth}|p{0.5in}|p{0.5in}|}
\cline{1-4}
\texttt{dm\_mech\_g1} & \texttt{Constant access time} & float & optional \\ 
\cline{1-4}
\multicolumn{4}{|p{6in}|}{
Provides the constant access time to be used if the access time type
is set to constant.
}\\ 
\cline{1-4}
\multicolumn{4}{p{5in}}{}\\
\end{tabular}\\ 
\noindent 
\begin{tabular}{|p{\lpmodwidth}|p{\lpnamewidth}|p{0.5in}|p{0.5in}|}
\cline{1-4}
\texttt{dm\_mech\_g1} & \texttt{Seek type} & string & required \\ 
\cline{1-4}
\multicolumn{4}{|p{6in}|}{
This specifies the method for computing seek delays.
Legal values are the following:
\texttt{linear} indicates that the single-cylinder seek time, the average
seek time, and the full-strobe seek time parameters should be used to
compute the seek time via linear interpolation.
\texttt{curve} indicates that the same three parameters should be used
with the seek equation described in \cite{Lee93} (see Section
\ref{seek.lee}).
\texttt{constant} indicates a fixed per-request seek time. The
\texttt{Constant seek time} parameter must be provided.
\texttt{hpl} indicates that the six-value \texttt{HPL seek equation values}
parameter (see below) should be used with the seek equation described
in \cite{Ruemmler94} (see below).
\texttt{hplplus10} indicates that the six-value \texttt{HPL seek
equation values} parameter (see below) should be used with the seek
equation described in \cite{Ruemmler94} for all seeks greater than
10~cylinders in length. For smaller seeks, use the 10-value
\texttt{First ten seek times} parameter (see below) as in
\cite{Worthington94}.
\texttt{extracted} indicates that a more complete seek curve (provided
in a separate file) should be used, with linear interpolation used to
compute the seek time for unspecified distances. If
\texttt{extracted} layout is used, the parameter \texttt{Full seek curve}
(below) must be provided.
}\\ 
\cline{1-4}
\multicolumn{4}{p{5in}}{}\\
\end{tabular}\\ 
\noindent 
\begin{tabular}{|p{\lpmodwidth}|p{\lpnamewidth}|p{0.5in}|p{0.5in}|}
\cline{1-4}
\texttt{dm\_mech\_g1} & \texttt{Average seek time} & float & optional \\ 
\cline{1-4}
\multicolumn{4}{|p{6in}|}{
The mean time necessary to perform a random seek
}\\ 
\cline{1-4}
\multicolumn{4}{p{5in}}{}\\
\end{tabular}\\ 
\noindent 
\begin{tabular}{|p{\lpmodwidth}|p{\lpnamewidth}|p{0.5in}|p{0.5in}|}
\cline{1-4}
\texttt{dm\_mech\_g1} & \texttt{Constant seek time} & float & optional \\ 
\cline{1-4}
\multicolumn{4}{|p{6in}|}{
For the ``constant'' seek type (above).
}\\ 
\cline{1-4}
\multicolumn{4}{p{5in}}{}\\
\end{tabular}\\ 
\noindent 
\begin{tabular}{|p{\lpmodwidth}|p{\lpnamewidth}|p{0.5in}|p{0.5in}|}
\cline{1-4}
\texttt{dm\_mech\_g1} & \texttt{Single cylinder seek time} & float & optional \\ 
\cline{1-4}
\multicolumn{4}{|p{6in}|}{
This specifies the time necessary to seek to an adjacent cylinder.
}\\ 
\cline{1-4}
\multicolumn{4}{p{5in}}{}\\
\end{tabular}\\ 
\noindent 
\begin{tabular}{|p{\lpmodwidth}|p{\lpnamewidth}|p{0.5in}|p{0.5in}|}
\cline{1-4}
\texttt{dm\_mech\_g1} & \texttt{Full strobe seek time} & float & optional \\ 
\cline{1-4}
\multicolumn{4}{|p{6in}|}{
This specifies the full-strobe seek time (i.e.,~the time to seek from the
innermost cylinder to the outermost cylinder).
}\\ 
\cline{1-4}
\multicolumn{4}{p{5in}}{}\\
\end{tabular}\\ 
\noindent 
\begin{tabular}{|p{\lpmodwidth}|p{\lpnamewidth}|p{0.5in}|p{0.5in}|}
\cline{1-4}
\texttt{dm\_mech\_g1} & \texttt{Full seek curve} & string & optional \\ 
\cline{1-4}
\multicolumn{4}{|p{6in}|}{
The name of the input file containing the seek curve data.
The format of this file is described below.
}\\ 
\cline{1-4}
\multicolumn{4}{p{5in}}{}\\
\end{tabular}\\ 
\noindent 
\begin{tabular}{|p{\lpmodwidth}|p{\lpnamewidth}|p{0.5in}|p{0.5in}|}
\cline{1-4}
\texttt{dm\_mech\_g1} & \texttt{Add. write settling delay} & float & required \\ 
\cline{1-4}
\multicolumn{4}{|p{6in}|}{
This specifies the additional time required to precisely settle the
read/write head for writing (after a seek or head switch). As this
parameter implies, the seek times computed using the above parameter
values are for read access.
}\\ 
\cline{1-4}
\multicolumn{4}{p{5in}}{}\\
\end{tabular}\\ 
\noindent 
\begin{tabular}{|p{\lpmodwidth}|p{\lpnamewidth}|p{0.5in}|p{0.5in}|}
\cline{1-4}
\texttt{dm\_mech\_g1} & \texttt{Head switch time} & float & required \\ 
\cline{1-4}
\multicolumn{4}{|p{6in}|}{
This specifies the time required for a head switch (i.e.,~activating a
different read/write head in order to access a different media
surface).
}\\ 
\cline{1-4}
\multicolumn{4}{p{5in}}{}\\
\end{tabular}\\ 
\noindent 
\begin{tabular}{|p{\lpmodwidth}|p{\lpnamewidth}|p{0.5in}|p{0.5in}|}
\cline{1-4}
\texttt{dm\_mech\_g1} & \texttt{Rotation speed (in rpms)} & int & required \\ 
\cline{1-4}
\multicolumn{4}{|p{6in}|}{
This specifies the rotation speed of the disk platters in rpms.
}\\ 
\cline{1-4}
\multicolumn{4}{p{5in}}{}\\
\end{tabular}\\ 
\noindent 
\begin{tabular}{|p{\lpmodwidth}|p{\lpnamewidth}|p{0.5in}|p{0.5in}|}
\cline{1-4}
\texttt{dm\_mech\_g1} & \texttt{Percent error in rpms} & float & required \\ 
\cline{1-4}
\multicolumn{4}{|p{6in}|}{
This specifies the maximum deviation in the rotation speed specified
above. During initialization, the rotation speed for each
disk is randomly chosen from a uniform distribution of the specified
rotation speed $\pm$ the maximum allowed error.
%(as computed from this parameter's value).
This feature may be deprecated and should be avoided.
}\\ 
\cline{1-4}
\multicolumn{4}{p{5in}}{}\\
\end{tabular}\\ 
\noindent 
\begin{tabular}{|p{\lpmodwidth}|p{\lpnamewidth}|p{0.5in}|p{0.5in}|}
\cline{1-4}
\texttt{dm\_mech\_g1} & \texttt{First ten seek times} & list & optional \\ 
\cline{1-4}
\multicolumn{4}{|p{6in}|}{
This is a list of ten floating-point numbers specifying the seek time for seek
distances of 1~through 10~cylinders.
}\\ 
\cline{1-4}
\multicolumn{4}{p{5in}}{}\\
\end{tabular}\\ 
\noindent 
\begin{tabular}{|p{\lpmodwidth}|p{\lpnamewidth}|p{0.5in}|p{0.5in}|}
\cline{1-4}
\texttt{dm\_mech\_g1} & \texttt{HPL seek equation values} & list & optional \\ 
\cline{1-4}
\multicolumn{4}{|p{6in}|}{
This is a list containing six numbers specifying the variables
$V_1$ through $V_6$ of the seek equation described in \cite{Ruemmler94}
(see below).
}\\ 
\cline{1-4}
\multicolumn{4}{p{5in}}{}\\
\end{tabular}\\ 


% -----------------------------------------------
% G2 Layout stuff

\subsubsection{G2 Layout}
% \subsubsection{dm\_layout\_g2}

The \texttt{dm\_layout\_g2} block provides parameters for a second
generation (g2) layout model.\\

\noindent 
\begin{tabular}{|p{\lpmodwidth}|p{\lpnamewidth}|p{0.5in}|p{0.5in}|}
\cline{1-4}
\texttt{dm\_layout\_g2} & \texttt{Layout Map File} & string & required \\ 
\cline{1-4}
\end{tabular}\\ 
\noindent 
\begin{tabular}{|p{\lpmodwidth}|p{\lpnamewidth}|p{0.5in}|p{0.5in}|}
\cline{1-4}
\texttt{dm\_layout\_g2} & \texttt{Zones} & list & required \\ 
\cline{1-4}
\end{tabular}\\ 


The \texttt{Zones} parameter is a list of zone blocks each of which
contains the following fields:\\

\noindent 
\begin{tabular}{|p{\lpmodwidth}|p{\lpnamewidth}|p{0.5in}|p{0.5in}|}
\cline{1-4}
\texttt{dm\_layout\_g2\_zone} & \texttt{First cylinder number} & int & required \\ 
\cline{1-4}
\multicolumn{4}{|p{6in}|}{
This specifies the first physical cylinder in the zone.
}\\ 
\cline{1-4}
\multicolumn{4}{p{5in}}{}\\
\end{tabular}\\ 
\noindent 
\begin{tabular}{|p{\lpmodwidth}|p{\lpnamewidth}|p{0.5in}|p{0.5in}|}
\cline{1-4}
\texttt{dm\_layout\_g2\_zone} & \texttt{Last cylinder number} & int & required \\ 
\cline{1-4}
\multicolumn{4}{|p{6in}|}{
This specifies the last physical cylinder in the zone.
}\\ 
\cline{1-4}
\multicolumn{4}{p{5in}}{}\\
\end{tabular}\\ 
\noindent 
\begin{tabular}{|p{\lpmodwidth}|p{\lpnamewidth}|p{0.5in}|p{0.5in}|}
\cline{1-4}
\texttt{dm\_layout\_g2\_zone} & \texttt{First LBN} & int & required \\ 
\cline{1-4}
\multicolumn{4}{|p{6in}|}{
The first LBN in this zone.
}\\ 
\cline{1-4}
\multicolumn{4}{p{5in}}{}\\
\end{tabular}\\ 
\noindent 
\begin{tabular}{|p{\lpmodwidth}|p{\lpnamewidth}|p{0.5in}|p{0.5in}|}
\cline{1-4}
\texttt{dm\_layout\_g2\_zone} & \texttt{Last LBN} & int & required \\ 
\cline{1-4}
\multicolumn{4}{|p{6in}|}{
The first LBN in this zone.
}\\ 
\cline{1-4}
\multicolumn{4}{p{5in}}{}\\
\end{tabular}\\ 
\noindent 
\begin{tabular}{|p{\lpmodwidth}|p{\lpnamewidth}|p{0.5in}|p{0.5in}|}
\cline{1-4}
\texttt{dm\_layout\_g2\_zone} & \texttt{Blocks per track} & int & required \\ 
\cline{1-4}
\multicolumn{4}{|p{6in}|}{
This specifies the number of sectors (independent of logical-to-physical
mappings) on each physical track in the zone.
}\\ 
\cline{1-4}
\multicolumn{4}{p{5in}}{}\\
\end{tabular}\\ 
\noindent 
\begin{tabular}{|p{\lpmodwidth}|p{\lpnamewidth}|p{0.5in}|p{0.5in}|}
\cline{1-4}
\texttt{dm\_layout\_g2\_zone} & \texttt{Zone Skew} & float & optional \\ 
\cline{1-4}
\multicolumn{4}{|p{6in}|}{
This specifies the physical offset of the first logical sector in the
zone. Physical sector 0 of every track is assumed to begin at the
same angle of rotation. This may be in either sectors or revolutions
according to the ''Skew units'' parameter.
}\\ 
\cline{1-4}
\multicolumn{4}{p{5in}}{}\\
\end{tabular}\\ 
\noindent 
\begin{tabular}{|p{\lpmodwidth}|p{\lpnamewidth}|p{0.5in}|p{0.5in}|}
\cline{1-4}
\texttt{dm\_layout\_g2\_zone} & \texttt{Skew units} & string & optional \\ 
\cline{1-4}
\multicolumn{4}{|p{6in}|}{
Default is \texttt{sectors}. This value overrides any set in the
surrounding layout block.
}\\ 
\cline{1-4}
\multicolumn{4}{p{5in}}{}\\
\end{tabular}\\ 
\noindent 
\begin{tabular}{|p{\lpmodwidth}|p{\lpnamewidth}|p{0.5in}|p{0.5in}|}
\cline{1-4}
\texttt{dm\_layout\_g2\_zone} & \texttt{Skew for track switch} & float & optional \\ 
\cline{1-4}
\multicolumn{4}{|p{6in}|}{
This specifies the number of physical sectors that are skipped when
assigning logical block numbers to physical sectors at a track
crossing point. Track skew is computed by the manufacturer to
optimize sequential access. This may be in either sectors or
revolutions according to the ''Skew units'' parameter.
}\\ 
\cline{1-4}
\multicolumn{4}{p{5in}}{}\\
\end{tabular}\\ 
\noindent 
\begin{tabular}{|p{\lpmodwidth}|p{\lpnamewidth}|p{0.5in}|p{0.5in}|}
\cline{1-4}
\texttt{dm\_layout\_g2\_zone} & \texttt{Skew for cylinder switch} & float & optional \\ 
\cline{1-4}
\multicolumn{4}{|p{6in}|}{
This specifies the number of physical sectors that are skipped when
assigning logical block numbers to physical sectors at a cylinder
crossing point. Cylinder skew is computed by the manufacturer to
optimize sequential access. This may be in either sectors or
revolutions according to the ''Skew units'' parameter.
}\\ 
\cline{1-4}
\multicolumn{4}{p{5in}}{}\\
\end{tabular}\\ 


% -----------------------------------------------
% G4 Layout stuff

\subsubsection{G4 Layout}
% \subsubsection{dm\_layout\_g4}

The \texttt{dm\_layout\_g4} block provides parameters for a fourth
generation (g4) layout model.\\

\noindent 
\begin{tabular}{|p{\lpmodwidth}|p{\lpnamewidth}|p{0.5in}|p{0.5in}|}
\cline{1-4}
\texttt{dm\_layout\_g4} & \texttt{TP} & list & required \\ 
\cline{1-4}
\multicolumn{4}{|p{6in}|}{
s0, -- lowest sector number on the track
sn, -- highest sector number on the track
spt -- physical spt on the track
Indicates a range of lbns [0,n] map to sectors [s0,sn] on a track.
This will almost always be [0,n-1,n] where n is the phsyical SPT.
}\\ 
\cline{1-4}
\multicolumn{4}{p{5in}}{}\\
\end{tabular}\\ 
\noindent 
\begin{tabular}{|p{\lpmodwidth}|p{\lpnamewidth}|p{0.5in}|p{0.5in}|}
\cline{1-4}
\texttt{dm\_layout\_g4} & \texttt{IDX} & list & required \\ 
\cline{1-4}
\multicolumn{4}{|p{6in}|}{
IDX contains a list of g4 layout index nodes which are themselves
represented as lists of integers. The inner lists contain the index
entries which have 8 fields as follows:
lbn, -- the lbn of the first instance, relative to the parent
cyl, -- the lowest cylinder in the first instance, relative to the parent
runlen, -- the number of lbns this entry covers
cylrunlen, -- the number of cylinders for this entry
len, -- the number of lbns in one instance of the child
cyllen, -- the number of cylinders covered by one instance
childtype, -- whether the child is an index node (IDX) or track pattern (TP)
child -- index of the child node in IDX or TP
The entries are given inline so the inner lists will have a multiple
of 8 entries.
The last index node (last entry in the outer list) is the "root" index
node that covers all of the LBN and cylinder space.
}\\ 
\cline{1-4}
\multicolumn{4}{p{5in}}{}\\
\end{tabular}\\ 
\noindent 
\begin{tabular}{|p{\lpmodwidth}|p{\lpnamewidth}|p{0.5in}|p{0.5in}|}
\cline{1-4}
\texttt{dm\_layout\_g4} & \texttt{Slips} & list & required \\ 
\cline{1-4}
\multicolumn{4}{|p{6in}|}{
A list of slip locations. Each slip location is described by 2
integers, the first being the logical offset (lbn) and the second
being the number of LBNs are slipped.
}\\ 
\cline{1-4}
\multicolumn{4}{p{5in}}{}\\
\end{tabular}\\ 
\noindent 
\begin{tabular}{|p{\lpmodwidth}|p{\lpnamewidth}|p{0.5in}|p{0.5in}|}
\cline{1-4}
\texttt{dm\_layout\_g4} & \texttt{Remaps} & list & required \\ 
\cline{1-4}
\multicolumn{4}{|p{6in}|}{
A list of remapped defects. Each remap is described by 6 integers:
LBN, -- logical offset of remapped location
len, -- number of contiguous sectors remapped
cylinder, head, sector, -- remap destination
spt -- physical SPT on destination track
}\\ 
\cline{1-4}
\multicolumn{4}{p{5in}}{}\\
\end{tabular}\\ 


\subsection{Seek Equation Definitions}


\subsubsection{Lee's Seek Equation}
\label{seek.lee}

\begin{math}
seekTime(x) = \left\{ \begin{array} {r@{\quad:\quad}l}
              0 & if x = 0 \\
              a\sqrt{x-1} + b(x-1) + c & if x > 0
              \end{array} \right., \mbox{where} \\
\\
\indent
x \quad\mbox{is the seek distance in cylinders,} \\
\indent
a = (-10 minSeek + 15 avgSeek - 5 maxSeek) / (3 \sqrt{numCyl}), \\
\indent
b = (7 minSeek - 15 avgSeek + 8 maxSeek) / (3 numCyl), \mbox{and}\\
\indent
c = minSeek.
\end{math}\\


\subsubsection{The HPL Seek Equation}
\label{seek.hpl}

\begin{tabular}{cc}
Seek distance     & Seek time \\ \hline
1 cylinder        & $V_6$ \\
$<$$V_1$ cylinders  & $V_2$ + $V_3$ * $\sqrt{dist}$ \\
$>=$$V_1$ cylinders & $V_4$ + $V_5$ * dist \\
\end{tabular}
, where {\it dist} is the seek distance in cylinders.
\newline
If $V_6 == -1$, single-cylinder seeks are computed using the second equation.
$V_1$ is specified in cylinders, and $V_2$ through $V_6$ are specified in 
milliseconds.

$V_1$ must be a non-negative integer, $V_2 \ldots V_5$ must be
non-negative floats and $V_6$ must be either a non-negative float or
$-1$.\\


\noindent\textbf{Format of an extracted seek curve}
\label{seek.extract}

An extracted seek file contains a number of (seek-time,seek-distance)
data points.
The format of such a file is very simple: the first line is
\[\texttt{Seek distances measured: <n>}\]
where \texttt{<n>} is the number of seek distances provided in the
curve.  This line is followed by \texttt{<n>} lines of the form
\texttt{<distance>, <time>} where \texttt{<distance>} is the seek
distance measured in cylinders, and \texttt{<time>} is the amount of
time the seek took in milliseconds. e.g.

\begin{verbatim}
Seek distances measured: 4
1,  1.2
2,  1.5
5,  5
10, 9.2
\end{verbatim}







\section{Installation}
To Build Diskmodel:

1.  build libparam and libtrace 

2.  edit .paths in the diskmodel source directory to reflect where
you built libparam and libtrace

3.  'make' in the diskmodel directory

\section{Typical use with libparam}

'make all' sets up include and lib subdirectories such that
you may use 
\begin{verbatim}-I$(DISKMODEL_PREFIX)/include \end{verbatim}
with the preprocessor and 
\begin{verbatim}#include<diskmodel/dm.h> \end{verbatim} etc.  Similarly, 
\begin{verbatim}-L$(DISKMODEL_PREFIX)/lib -ldiskmodel \end{verbatim} 
with the linker where \texttt{DISKMODEL\_PREFIX} is the top-level
source directory where you built diskmodel.


1.  register diskmodel libparam modules with libparam. e.g.
\begin{verbatim}
#include <diskmodel/modules/modules.h>
for(i = 0; i <= DM_MAX_MODULE; i++) {
  lp_register_module(dm_mods[i]);
}
\end{verbatim}

2.  use \texttt{lp\_loadfile()} to load a model file

3.  use \texttt{lp\_instantiate()} to instantiate a model from the
input file.  The result of the instantiation is a \texttt{struct
dm\_disk\_if *}

e.g. \texttt{struct dm\_disk\_if *disk = lp\_instantiate(...);}


4.  Access methods through d.
e.g. \texttt{dm\_time\_t seektime = d->mech->dm\_seek\_time(...)}


\begin{thebibliography}{99}   

%\bibitem [Abbott90] {Abbott90}
%R. Abbott, H. Garcia-Molina,
%``Scheduling I/O Request with Deadlines: a Performance Evaluation'',
%{\it IEEE Real-Time Systems Symposium},
%1990, \mbox{pp. 113--124}.

%\bibitem [Amdahl67] {Amdahl67}
%G. Amdahl,
%``Validity of the single processor approach to achieving large scale computing capabilities'',
%{\it AFIPS Spring Joint Computing Conference},
%\mbox{Vol. 30}, April 1967, pp. 483--485.

%\bibitem [Aschoff89] {Aschoff89}
%J. Aschoff,
%``Performance Management of Storage Subsystems'',
%{\it Computer Measurement Group (CMG) Conference},
%1989, pp. 730--739.

%\bibitem [Baker91] {Baker91}
%M. Baker, J. Hartman, M. Kupfer, K. Shirriff, J. Ousterhout,
%``Measurements of a Distributed File System'',
%{\it ACM Symposium on Operating Systems Principles},
%1991, pp. 198--212.

%\bibitem [Baker92] {Baker92}
%M. Baker, S. Asami, E. Deprit, J. Ousterhout, M. Seltzer,
%``Non-Volatile Memory for Fast, Reliable File Systems'',
%{\it ACM ASPLOS Conference},
%October 1992, pp. 10--22.

%\bibitem [Bates91] {Bates91}
%K. Bates,
%{\it VAX I/O Subsystems: Optimizing Performance},
%Professional Press Books, 1991.

%\bibitem [Bates91a] {Bates91a}
%K. Bates,
%``I/O Subsystem Performance'',
%{\it DEC Professional},
%November 1991, pp. 60--70.

%\bibitem [Bates92] {Bates92}
%K. Bates,
%``I/O Subsystem Performance'',
%{\it DEC Professional},
%February 1992, pp. 42--49.

%\bibitem [Bell84] {Bell84}
%C. Bell,
%``The Mini and Micro Industries'',
%{\it IEEE Computer},
%Vol. 17, pp. 14--30, October 1984.

%\bibitem [Bennett94] {Bennett94}
%S. Bennett, D. Melski,
%``A Class-Based Disk Scheduling Algorithms:  Implementation and 
%Performance Study'',
%Unpublished Report, University of Wisconsin, Madison,
%1994.

%\bibitem [Berry91] {Berry91}
%R. Berry, J. Hellerstain, J. Kolb, P. VanLeer,
%``Choosing a Service Level Indicator: Why Not Queue Length?'',
%{\it Computer Measurement Group (CMG) Conference},
%1991, pp. 404--413.

%\bibitem [Bershad95] {Bershad95}
%B. Bershad, S. Savage, et al.,
%``Extensiblity, Safety and Performance in the SPIN Operating System'',
%{\it ACM Symposium on Operating System Principles},
%Dec. 1995, pp. 267-284.

%\bibitem [Bhide93] {Bhide93}
%A. Bhide, D. Dias, N. Halim, B. Smith, F. Parr,
%``A Case for Fault-Tolerant Memory for Transaction Processing'',
%{\it IEEE FTCS Conference},
%1993, pp. 451--460.

%\bibitem [Biswas93] {Biswas93}
%P. Biswas, K.K. Ramakrishnan, D. Towsley,
%``Trace Driven Analysis of Write Caching Policies for Disks'',
%{\it ACM SIGMETRICS International Conference on Measurement and Modeling of Computer Systems},
%May 1993, pp. 13--23.

%\bibitem [Bitton88] {Bitton88}
%D. Bitton, J. Gray,
%``Disk Shadowing'',
%{\it International Conference on Very Large Data Bases},
%September 1988, pp. 331--338.

%\bibitem [Bitton89] {Bitton89}
%D. Bitton,
%``Arm Scheduling in Shadowed Disks'',
%{\it IEEE COMPCON},
%Spring 1989, pp. 132--136.

%\bibitem [Blackwell95] {Blackwell95}
%T. Blackwell, J. Harris, M. Seltzer,
%``Heuristic Cleaning Algorithms in Log-Structured File Systems'',
%{\it USENIX Technical Conference},
%January 1995, pp. 277-288.

%\bibitem [Bozm91] {Bozman91}
%G. Bozman, H. Ghannad, E. Weinberger,
%``A Trace-Driven Study of CMS File References'',
%{\it IBM Journal of Research and Development},
%Vol. 35, Vol. 5/6, September/November 1991, pp. 815--828.

%\bibitem [Brandwajn94] {Brandwajn94}
%A. Brandwajn, D. Levy,
%``A Study of Cached RAID-5 I/O'',
%{\it Computer Measurement Group (CMG) Conference},
%1994, pp. 393--403.

%\bibitem [Busch85] {Busch85}
%J. Busch, A. Kondoff,
%``Disc Caching in the System Processing Units of the HP 3000 Family of Computers",
%{\it HP Journal},
%Vol. 36, No. 2, February 1985, pp. 21--39.

%\bibitem [Buzen90] {Buzen90}
%J. Buzen, A. Shum,
%``I/O Performance Trade-Offs and MVS/ESA Considerations'',
%{\it Computer Measurement Group (CMG) Conference},
%1990, pp. 695--702.

%\bibitem [Cao93] {Cao93}
%P. Cao, S. Lim, S. Venkataraman, J. Wilkes,
%``The TickerTAIP Parallel RAID Architecture'',
%{\it IEEE International Symposium on Computer Architecture},
%May 1993, pp. 52--63.

%\bibitem [Cao94] {Cao94}
%P. Cao, E. Felten, K. Li,
%``Implementation and Performance of Application-Controlled File Caching'',
%{\it Symposium on Operating Systems Design and Implementation (OSDI)},
%November 1994, pp. 179--194.

%\bibitem [Cao95] {Cao95}
%P. Cao, E. Felton, A. Karlin, K. Li,
%``A Study of Integrated Prefetching and Caching Strategies'',
%{\it ACM SIGMETRICS International Conference on Measurement and Modeling of Computer Systems},
%May 1995, \mbox{pp. 188-197}.

%\bibitem [Carey89] {Carey89}
%M. Carey, R. Jauhari, M. Livny,
%``Priority in DBMS Resource Scheduling'',
%{\it International Conference on Very Large Data Bases},
%1989, pp. 397--410.

%\bibitem [Carson92] {Carson92}
%S. Carson, S. Setia,
%``Analysis of the Periodic Update Write Policy for Disk Cache'',
%{\it IEEE Transactions on Software Engineering},
%Vol. 18, No. 1, January 1992, pp. 44--54.

%\bibitem [Carson92a] {Carson92a}
%S. Carson, S. Setia,
%``Optimal Write Batch Size in Log-Structured File Systems'',
%{\it USENIX File Systems Workshop},
%May 1992, pp. 79--91.

%\bibitem [Chamberlin81] {Chamberlin81}
%D. Chamberlin, M. Astrahan, et. al.,
%``A History and Evaluation of System R'',
%{\it Communications of the ACM},
%Vol. 24, No. 10, 1981, pp. 632--646.

%\bibitem [Chao92] {Chao92}
%C. Chao, R. English, D. Jacobson, A. Stepanov, J. Wilkes,
%``Mime: A High-Performance Parallel Storage Device with Strong Recovery
%Guarantees'',
%Hewlett-Packard Laboratories Report, HPL-CSP-92-9 rev 1,
%November 1992.

%\bibitem [Chen90] {Chen90}
%P. Chen, D. Patterson,
%``Maximizing throughput in a striped disk array'',
%{\it IEEE International Symposium on Computer Architecture},
%1990, pp. 322--331.

%\bibitem [Chen90a] {Chen90a}
%P. Chen, G. Gibson, R. Katz, D. Patterson,
%``An Evaluation of Redundant Arrays of Disks using an Amdahl 5890'',
%{\it ACM SIGMETRICS International Conference on Measurement and Modeling of Computer Systems},
%May 1990, \mbox{pp. 74--85}.

%\bibitem [Chen91] {Chen91}
%S. Chen, D. Towsley,
%``A Queueing Analysis of RAID Architectures'',
%University of Massachusetts, Amherst, COINS Technical Report 91--71,
%September 1991.

%\bibitem [Chen91a] {Chen91a}
%S. Chen, J. Kurose, J. Stankovic, D. Towsley,
%``Performance Evaluation of Two New Disk Request Scheduling Algorithms for Real-Time Systems'',
%{\it Journal of Real-Time Systems},
%Vol. 3, 1991, pp. 307--336.

%\bibitem [Chen92] {Chen92}
%P. Chen,
%``Input-Output Performance Evaluation: Self-Scaling Benchmarks, Predicted Performance'',
%Ph.D. Dissertation, University of California at Berkeley,
%1992.

%\bibitem [Chen93] {Chen93}
%P. Chen, D. Patterson,
%``A New Approach to I/O Performance Evaluation -- Self-Scaling
%I/O Benchmarks, Predicted I/O Performance'',
%{\it ACM SIGMETRICS International Conference on Measurement and Modeling of Computer Systems},
%May 1993, pp. 1--12.

%\bibitem [Chen93a] {Chen93a}
%P. Chen, E. Lee, A. Drapeau, K. Lutz, E. Miller, S. Seshan, K. Sherriff,
%D. Patterson, R. Katz,
%``Performance and Design Evaluation of the RAID-II Storage Server'',
%{\it International Parallel Processing Symposium, Workshop on I/O},
%1993.

%\bibitem [Chen93b] {Chen93b}
%P. Chen, E. Lee, G. Gibson, R. Katz, D. Patterson,
%``RAID: High-Performance, Reliable Secondary Storage'',
%{\it ACM Computing Surveys},
%Vol. 26, No. 2, June 1994, pp. 145--188.
%%Technical Report No. UCB/CSD-93-778, University of California, Berkeley,
%%October 29, 1993.

%\bibitem [Chen95] {Chen95}
%P. Chen, E. Lee,
%``Striping in a RAID Level 5 Disk Array'',
%{\it ACM SIGMETRICS International Conference on Measurement and Modeling of Computer Systems},
%May 1995, pp. 136--145.

%\bibitem [Cheriton94] {Cheriton94}
%D. Cheriton, K. Duda,
%``A Caching Model of Operating System Kernel Functionality'',
%{\it USENIX Symposium on Operating System Design and Implementation},
%Nov. 1994, pp. 179-193.

%\bibitem [Chervenak90] {Chervenak90}
%A.L. Chervenak,
%``Performance Measurements of the First RAID Prototype'',
%U.C. Berkeley Technical Report UCB/CSD 90/475,
%May 1990.

%\bibitem [Chervenak91] {Chervenak91}
%A.L. Chervenak, R. Katz,
%``Performance of a RAID Prototype'',
%{\it ACM SIGMETRICS International Conference on Measurement and Modeling of Computer Systems},
%May 1991, pp. 188--197.

%\bibitem [Chiu78] {Chiu78}
%W. Chiu, W. Chow,
%``A Performance Model of MVS'',
%{\it IBM System Journal},
%Vol. 17, No. 4, 1978, pp. 444--463.

%\bibitem [Chutani92] {Chutani92}
%S. Chutani, O. Anderson, M. Kazar, B. Leverett, W. Mason, R. Sidebotham,
%``The Episode File System'',
%{\it Winter USENIX Conference},
%January 1992, pp. 43--60.

%\bibitem [Clark90] {Clark90}
%D. Clark, D. Tennenhouse,
%``Architectural Considerations for a New Generation of Protocols'',
%{\it ACM SIGCOMM Conference}, Sept. 1990.

%\bibitem [Coff72] {Coffman72}
%E. G. Coffman, L. A. Klimko, B. Ryan,
%``Analysis of Scanning Policies for Reducing Disk Seek Times'',
%{\it SIAM Journal of Computing},
%Vol. 1, No. 3, September 1972, \mbox{pp. 269--279}.

%\bibitem [Coffman82] {Coffman82}
%E. Coffman, M. Hofri,
%``On the Expected Performance of Scanning Disks'',
%{\it SIAM Journal of Computing},
%Vol. 11, No. 1, February 1982, \mbox{pp. 60--70}.

%\bibitem [Copeland89] {Copeland89}
%G. Copeland, T. Keller,
%``A Comparison of High-Availability Media Recovery Techniques'',
%{\it ACM SIGMOD International Conference on Management of Data},
%1989, pp. 98--109.

%\bibitem [Crockett89] {Crockett89}
%T. Crockett,
%``File Concepts for Parallel I/O'',
%{\it Proceedings of Supercomputing},
%1989, pp. 574--579.

%\bibitem [Dahlin94] {Dahlin94}
%M. Dahlin, et alia,
%``Cooperative Caching: Using Remote Client Memory to Improve File System Performance'',
%{\it Symposium on Operating Systems Design and Implementation (OSDI)},
%November 1994, pp. 267--280.

%\bibitem [Dan94] {Dan94}
%A. Dan, D. Dias, P. Yu,
%``Buffer Analysis for a Data Sharing Environment with Skewed Data Access'',
%{\it IEEE Transactions on Knowledge and Data Engineering},
%Vol. 6, No. 2, April 1994, pp. 331--337.

%\bibitem [Daniel83] {Daniel83}
%S. Daniel, R. Geist,
%``V-SCAN:  An adaptive disk scheduling algorithm'',
%{\it IEEE International Workshop on Computer Systems Organization},
%%New Orleans, Louisiana,
%March 1983, \mbox{pp. 96--103}.

%\bibitem [DDI90] {DDI90}
%{\it Device Driver Interface/Driver-Kernel Interface (DDI/DKI) Reference
%Manual},
%UNIX System V/386 Release 4, AT\&T, 1990.

%\bibitem [Denning67] {Denning67}
%P. J. Denning,
%``Effects of scheduling on file memory operations'',
%{\it AFIPS Spring Joint Computer Conference},
%%Atlantic City, New Jersey,
%April 1967, pp. 9--21.

%\bibitem [Denn68] {Denning68}
%P.J. Denning,
%``The Working Set Model for Program Behavior'',
%{\it Communications of the ACM},
%Vol. 11, No. 5, May 1968, pp. 323--333.

%\bibitem [Dibble88] {Dibble88}
%P. Dibble, M. Scott, C. Ellis,
%``Bridge: A High Performance File System for Parallel Processors'',
%{\it 8th International Conference on Distributed Computing Systems},
%pp. 154--161.

%\bibitem [Dibble89] {Dibble89}
%P. Dibble, M. Scott,
%``Beyond Striping: The Bridge Multiprocessor File System'',
%{\it ACM Computer Architecture News},
%September 1989, pp. 32--39.

%\bibitem [Digital92] {Digital92}
%{\it Digital Storage Technology Handbook},
%Digital Equipement Corporation, Northboro, MA, 1992.

%\bibitem [Ebling94] {Ebling94}
%M. Ebling, M. Satyanarayanan,
%``SynRGen: An Extensible File Reference Generator'',
%{\it ACM SIGMETRICS International Conference on Measurement and Modeling of Computer Systems},
%May 1994, pp. 108--117.

%\bibitem [Effe84] {Effelsberg84}
%W. Effelsberg, M. Loomis,
%``Logical, Internal and Physical Reference Behavior in CODASYL Database Systems'',
%{\it ACM Transactions on Database Systems},
%Vol. 9, No. 2, June 1984, pp. 187--213.

%\bibitem [Engler95] {Engler95}
%D. Engler, M.F. Kaashoek, J. O'Toole Jr.,
%``Exokernel: an operating system architecture for application-level resource management'',
%{\it ACM Symposium on Operating Systems Principles},
%Dec. 1995, pp. 251-266.

%\bibitem [English91] {English91}
%R. English, A. Stepanov,
%``Loge: A Self-Organizing Disk Controller'',
%Hewlett-Packard Laboratories Report, HPL-91-179,
%December 1991.

%\bibitem [Ferr84] {Ferrari84}
%D. Ferrari,
%``On the Foundation of Artificial Workload Design'',
%{\it ACM SIGMETRICS International Conference on Measurement and Modeling of Computer Systems},
%May 1984, \mbox{pp. 8-14}.

%\bibitem [Forin94] {Forin94}
%A. Forin, G. Malan,
%``An MS-DOS File System for UNIX'',
%{\it Winter USENIX Conference},
%January 1994, pp. 337-354.

%\bibitem [Fujitsu90] {Fujitsu90}
%Fujitsu Limited,
%``M2622Sx/M2623Sx/M2624Sx Intelligent Disk Drives OEM Manual -- Specifications
%\& Installation'',
%Specification Number 41FH5055E-01, December 1990.

%\bibitem [Fujitsu91] {Fujitsu91}
%Fujitsu Limited,
%``M262X Product Brochure'',
%January 1991.

%\bibitem [Fujitsu91a] {Fujitsu91a}
%Fujitsu Limited,
%``M2652 Product Brochure'',
%1991.

%\bibitem [Gaede81] {Gaede81}
%S. Gaede,
%``Tools for Research in Computer Workload Characterization'',
%{\it Experimental Computer Performance and Evaluation},
%1981, ed. by D. Ferrari and M. Spadoni.

%\bibitem [Gaede82] {Gaede82}
%S. Gaede,
%``A Scaling Technique for Comparing Interactive System Capacities'',
%{\it 13th International Conference on Management and Performance Evaluation
%of Computer Systems},
%1982, \mbox{pp. 62--67}.

\bibitem [Ganger93] {Ganger93}
G. Ganger, Y. Patt,
``The Process-Flow Model: Examining I/O Performance from the System's Point of View'',
{\it ACM SIGMETRICS Conference},
May 1993, \mbox{pp. 86--97}.

\bibitem [Ganger93a] {Ganger93a}
G. Ganger, B. Worthington, R. Hou, Y. Patt,
``Disk Subsystem Load Balancing: Disk Striping vs. Conventional Data Placement'',
{\it Hawaii International Conference on System Sciences},
January 1993, pp. 40--49.

%\bibitem [Ganger94] {Ganger94}
%G. Ganger, Y. Patt,
%``Metadata Update Performance in File Systems'',
%{\it USENIX Symposium on Operating Systems Design and Implementation (OSDI)},
%November 1994, pp. 49--60.

\bibitem [Ganger94] {Ganger94}
G. Ganger, B. Worthington, R. Hou, Y. Patt,
``Disk Arrays: High Performance, High Reliability Storage Subsystems'',
{\it IEEE Computer},
Vol. 27, \mbox{No. 3}, March 1994, pp. 30--36.

%\bibitem [Ganger95] {Ganger95}
%G. Ganger, Y. Patt,
%``Soft Updates: A Solution to the Metadata Update Problem in File Systems'',
%Report CSE-TR-254-95, University of Michigan, Ann Arbor,
%August 1995.

\bibitem [Ganger95] {Ganger95}
G. Ganger,
``System-Oriented Evaluation of Storage Subsystem Performance'',
Ph.D. Dissertation, CSE-TR-243-95, University of Michigan, Ann Arbor,
June 1995.

\bibitem [Ganger95a] {Ganger95a}
G. Ganger,
``Generating Representative Synthetic Workloads An Unsolved Problem'',
{\it Computer Measurement Group (CMG) Conference},
Decemeber 1995, pp. 1263--1269.

\bibitem [Ganger98] {Ganger98}
G. Ganger, B. Worthington, Y. Patt,
``The DiskSim Simulation Environment Version 1.0 Reference Manual'',
Technical Report CSE-TR-358-98, University of Michigan, Ann Arbor,
February 1998.

\bibitem [Ganger99] {Ganger99}
G. Ganger, B. Worthington, Y. Patt,
``The DiskSim Simulation Environment Version 2.0 Reference Manual'',
December 1999.

%\bibitem [Geist87] {Geist87}
%R. Geist, S. Daniel,
%``A Continuum of Disk Scheduling Algorithms'',
%{\it ACM Transactions on Computer Systems},
%Vol. 5, No. 1, February 1987, pp. 77--92.

%\bibitem [Geist87a] {Geist87a}
%R. Geist, R. Reynolds, E. Pittard,
%``Disk Scheduling in System V'',
%%{\it Performance Evaluation Review},
%{\it ACM SIGMETRICS International Conference on Measurement and Modeling of Computer Systems},
%May 1987, pp. 59--68.

%\bibitem [Geist94] {Geist94}
%R. Geist, J. Westall,
%``Disk Scheduling in Linux'',
%{\it Computer Measurement Group (CMG) Conference},
%December 1994, pp. 739--746.

%\bibitem [Gibson89] {Gibson89}
%G. Gibson, L. Hellerstein, R. Karp, R. Katz, D. Patterson,
%``Failure Correction Techniques for Large Disk Arrays'',
%{\it 3rd ASPLOS},
%1989, pp. 123--132.

%\bibitem [Gingell87] {Gingell87}
%R. Gingell, J. Moran, W. Shannon,
%``Virtual Memory Architecture in SunOS'',
%{\it Summer USENIX Conference},
%June 1987, \mbox{pp. 81--94}.

%\bibitem [Golding95] {Golding95}
%R. Golding, P. Bosch, C. Staelin, T. Sullivan, J. Wilkes,
%``Idleness is not sloth'',
%{\it Winter USENIX Conference},
%January 1995, pp. 201--22.

%\bibitem [Gotl73] {Gotlieb73}
%C. C. Gotlieb, G. H. MacEwen,
%``Performance of Movable-Head Disk Storage Devices'',
%{\it Journal of the Association for Computing Machinery},
%Vol. 20, No. 4, October 1973, pp. 604--623.

%\bibitem [Gray90] {Gray90}
%J. Gray, B. Horst, M. Walker,
%``Parity Striping of Disk Arrays:
%Low-Cost Reliable Storage with Acceptable Throughput'',
%{\it Very Large DataBases Conference},
%August 1990, pp. 148--161.

%\bibitem [Gray91] {Gray91}
%ed. J. Gray,
%{\it The Benchmark Handbook for Database and Transaction Processing Systems},
%Morgan Kaufman Publishers, San Mateo, CA,
%1991.

%\bibitem [Gray93] {Gray93}
%Jim Gray, Andrea Reuter,
%{\it Transaction Processing:  Concepts and Techniques}
%Morgan Kaufmann Publishers, San Mateo, CA,
%1993.

%\bibitem [Griffioen94] {Griffioen94}
%J. Griffioen, R. Appleton,
%``Reducing File System Latency using a Predictive Approach'',
%{\it Summer USENIX Conference},
%June 1994, \mbox{pp. 197--207}.

%\bibitem [Grossman85] {Grossman85}
%C.P.Grossman,
%``Cache-DASD Storage Design for Improving System Performance'',
%{\it IBM Systems Journal},
%Vol. 24, No. 3/4,
%1985, pp. 316--334.

%\bibitem [Hagmann87] {Hagmann87}
%R. Hagmann,
%``Reimplementing the Cedar File System Using Logging and Group Commit'',
%{\it ACM Symposium on Operating Systems Principles},
%November 1987, pp. 155--162.
%%published by ACM as {\it Operating Systems Review}, Vol. 21, No. 5,

%\bibitem [Haigh90] {Haigh90}
%P. Haigh,
%``An Event Tracing Method for UNIX Performance Measurement'',
%{\it Computer Measurement Group (CMG) Conference},
%1990, pp. 603--609.

%\bibitem [Harker81] {Harker81}
%J.M. Harker, D.W. Brede, R.E. Pattison, G.R. Santana, L.G. Taft,
%``A Quarter Century of Disk File Innovation'',
%{\it IBM Journal of Research and Development},
%September 1981, pp. 677--689.

%\bibitem [Hartman94] {Hartman94}
%J. Hartman, A. Montz, et al.,
%``Scout: A Communication-Oriented Operating System'',
%Technical Report TR 94-20, University of Arizona, Tucson, AZ,
%June 1994.

%\bibitem [Henl89] {Henley89}
%M. Henley, B. McNutt,
%``DASD I/O Characteristics, A Comparison of MVS to VM'',
%IBM Technical Report, TR 02.1550,
%May 1989.

%\bibitem [Hennessy90] {Hennessy90}
%J. Hennessy, D. Patterson,
%{\it Computer Architecture: A Quantitative Approach},
%Morgan Kaufmann Publishers, Inc., San Mateo, California,
%1990.

%\bibitem [Hitz95] {Hitz95}
%D. Hitz,
%``An NFS File Server Appliance'',
%Technical Report 3001, Network Appliance Corporation,
%March 1995.

\bibitem [Holland92] {Holland92}
M. Holland, G. Gibson,
``Parity Declustering for Continuous Operation in Redundant Disk Arrays'',
{\it ACM International Conference on Architectural Support for Programming Languages and Operating Systems},
October 1992, \mbox{pp. 23--35}.

%\bibitem [Hofri80] {Hofri80}
%M. Hofri,
%``Disk Scheduling: FCFS vs. SSTF Revisited'',
%{\it Communications of the ACM},
%Vol. 23, No. 11, November 1980, pp. 645--653.

%\bibitem [Hospodor94] {Hospodor94}
%A. Hospodor,
%``The Effect of Prefetch in Caching Disk Buffers'',
%Ph.D. Dissertation, Santa Clara University,
%1994.

%\bibitem [Hou92] {Hou92}
%R. Hou, G. Ganger, Y. Patt, C. Gimarc,
%``Issues and Problems in the I/O Subsystem, Part I -- The Magnetic Disk'',
%to be presented at the {\it Hawaii International Conference on System Sciences},
%January 1992.

%\bibitem [Hou93] {Hou93}
%R. Hou, Y. Patt,
%``Comparing Rebuild Algorithms for Mirrored and RAID5 Disk Arrays'',
%{\it ACM SIGMOD International Conference on Management of Data},
%May 1993, pp. 317--326.

%\bibitem [Hou93a] {Hou93a}
%R. Hou, J. Menon, Y. Patt,
%``Balancing I/O Response Time and Disk Rebuild Time in a RAID5 Disk Array'',
%{\it Hawaii International Conference on System Sciences},
%January 1993, pp. 70--79.

%\bibitem [Hou93b] {Hou93b}
%R. Hou, Y. Patt,
%``Trading Disk Capacity for Performance'',
%{\it International Symposium on High-Performance Distributed Computing},
%July 1993, \mbox{pp. 263--270}.

%\bibitem [Houtekamer85] {Houtekamer85}
%G. Houtekamer,
%``The Local Disk Controller'',
%{\it ACM SIGMETRICS International Conference on Measurement and Modeling of Computer Systems},
%May 1985, pp. 173--182.

%\bibitem [Howard88] {Howard88}
%J. Howard, M. Kazar, S. Menees, D. Nichols, M. Satyanarayanan, R. Sidebotham, M. West,
%``Scale and Performance in a Distributed File System'',
%{\it IEEE Transactions on Computer Systems},
%Vol. 6, No. 1, February 1988, pp. 51--81.

%\bibitem [HP90] {HP90}
%Hewlett-Packard Company,
%``HP 97556/58/60 5.25-inch SCSI Disk Drives -- Product Description Manual'',
%Draft Edition 3, December 1990.

\bibitem [HP91] {HP91}
Hewlett-Packard Company,
``HP C2247 3.5-inch SCSI-2 Disk Drive -- Technical Reference Manual'',
Edition 1, Draft, December 1991.

\bibitem [HP92] {HP92}
Hewlett-Packard Company,
``HP C2244/45/46/47 3.5-inch SCSI-2 Disk Drive Technical Reference Manual'',
Part Number 5960-8346, Edition 3, September 1992.

\bibitem [HP93] {HP93}
Hewlett-Packard Company,
``HP C2490A 3.5-inch SCSI-2 Disk Drives, Technical Reference Manual'',
Part Number 5961-4359, Edition 3, September 1993.

\bibitem [HP94] {HP94}
Hewlett-Packard Company,
``HP C3323A 3.5-inch SCSI-2 Disk Drives, Technical Reference Manual'',
Part Number 5962-6452, Edition 2, April 1994.

%\bibitem [HP96] {HP96}
%Hewlett-Packard Company,
%http://www.dmo.hp.com/disks/oemdisk/c3653a.html,
%June 1996.

%\bibitem [Hsiao90] {Hsiao90}
%H. Hsiao, D. DeWitt,
%``Chained Declustering: A New Availability Strategy for Multiprocessor
%Database Machines'',
%{\it IEEE International Conference on Data Engineering},
%1990, pp. 456--465.

%\bibitem [Hunter80] {Hunter80}
%D. Hunter,
%``Modeling Real DASD Configurations'',
%IBM Research Report, RC 8606, September 1981, pp. 677--689.

%\bibitem [Jacobson91] {Jacobson91}
%D. Jacobson, J. Wilkes,
%``Disk Scheduling Algorithms Based on Rotational Position'',
%Hewlett-Packard Technical Report, HPL-CSP-91-7,
%February~26, 1991.

%\bibitem [Jaffe93] {Jaffe93}
%D. Jaffe,
%``Architecture of a Fault-Tolerant RAID-5+ I/O Subsystem'',
%{\it Hawaii International Conference on System Sciences},
%January 1993, pp. 60--69.

%\bibitem [Jones89] {Jones89}
%A. Jones,
%``How Does Your Garden Grow?'',
%{\it Computer Measurement Group (CMG) Conference},
%1989, pp. 740--748.

%\bibitem [Journal92] {Journal92}
%NCR Corporation,
%``Journaling File System Administrator Guide, Release 2.00'',
%NCR Document D1-2724-A,
%April 1992.

\bibitem [Karedla94] {Karedla94}
R. Karedla, J. S. Love, B. Wherry,
``Caching Strategies to Improve Disk System Performance'',
{\it IEEE Computer},
Vol. 27, No. 3, March 1994, pp. 38--46.

%\bibitem [Katz89] {Katz89}
%R.H. Katz, G.A. Gibson, D.A. Patterson,
%``Disk System Architectures for High Performance Computing'',
%{\it Proceedings of the IEEE},
%Vol. 77, No. 12, December 1989, pp. 1842--1858.

%\bibitem [Kearns89] {Kearns89}
%J. Kearns, S. DeFazio,
%``Diversity in Database Reference Behavior'',
%{\it SIGMETRICS},
%May 1989, pp. 11--19.

%\bibitem [Kim85] {Kim85}
%M. Y. Kim, A. M. Patel,
%``Error-Correcting Codes for Interleaved Disks with Minimal Redundancy'',
%IBM Research Report, RC 11185, May 31, 1985.

%\bibitem [Kim85] {Kim85}
%M. Kim,
%``Parallel Operation of Magnetic Disk Storage Devices: Synchronized Disk Interleaving'',
%{\it International Workshop on Database Machines},
%March 1985, pp. 300--330.

%\bibitem [Kim86] {Kim86}
%M. Kim,
%``Synchronized Disk Interleaving'',
%{\it IEEE Transactions on Computers},
%Vol. C-35, No. 11, November 1986, pp. 978--988.

%\bibitem [Kim91] {Kim91}
%M. Kim,
%``Asynchronous Disk Interleaving: Approximating Access Delays'',
%{\it IEEE Transactions on Computers},
%Vol. 40, No. 7, July 1991, pp. 801--810.

%\bibitem [Kleiman86] {Kleiman86}
%S. Kleiman,
%``Vnodes: An Architecture for Multiple File System Types in Sun UNIX'',
%{\it Summer USENIX Conference},
%1986.

%\bibitem [Kondoff88] {Kondoff88}
%A. Kondoff,
%"The MPE XL Data Management System: Exploiting the HP Precision
%Architecture for HP's Next Generation Commercial Computer Systems",
%{\it IEEE CompCon},
%1988, pp. 152--155.

%\bibitem [Kotz94] {Kotz94}
%D. Kotz, S. Toh, S. Radhakrishnan,
%``A Detailed Simulation Model of the HP 97560 Disk Drive'',
%Report No. PCS-TR94-220, Dartmouth College,
%July 18, 1994.

%\bibitem [Lary93] {Lary93}
%R. Lary, Storage Architect, Digital Equipment Corporation,
%Personal Communication, February 1993.

%\bibitem [Lee90] {Lee90}
%E. Lee,
%``Software and Performance Issues in the Implementation of a RAID Prototype'',
%Report No. UCB/CSD 90/573, University of California, Berkeley,
%May 1990.

\bibitem [Lee91] {Lee91}
E. Lee, R. Katz,
``Peformance Consequences of Parity Placement in Disk Arrays'',
{\it ACM International Conference on Architectural Support for
Programming Languages and Operating Systems},
1991, pp. 190--199.

\bibitem [Lee93] {Lee93}
E. Lee, R. Katz,
``An Analytic Performance Model of Disk Arrays'',
{\it ACM Sigmetrics Conference},
May 1993, pp. 98-109.

%\bibitem [Lee93] {Lee93}
%E. Lee,
%``Performance Modeling and Analysis of Disk Arrays'',
%Ph.D. Dissertation, University of California, Berkeley,
%1993.

%\bibitem [Leland93] {Leland93}
%W. Leland, M. Taqqu, W. Willinger, D. Wilson,
%``On the Self-Similar Nature of Ethernet Traffic'',
%{\it ACM SIGCOMM Conference},
%1993, pp. 183--193.

%\bibitem [Lim91] {Lim91}
%S.B. Lim, M. Condry,
%``Supercomputing Application Access Characteristics'',
%Technical Report UIUCDCS-R-91-1708, University of Illinois, Urbana-Champaigne,
%October 1991.

%\bibitem [Livny87] {Livny87}
%M. Livny, S. Khoshafian, H. Boral,
%``Multi-Disk Management Algorithms'',
%{\it ACM SIGMETRICS International Conference on Measurement and Modeling of Computer Systems},
%May 1987, pp. 69--77.

%\bibitem [Major87] {Major87}
%J. Major,
%``Empirical Models of DASD Response Time'',
%{\it Computer Measurement Group (CMG) Conference},
%1987, pp. 390--398.

%\bibitem [Major94] {Major94}
%D. Major, G. Minshall, K. Powell,
%``An Overview of the NetWare Operating System'',
%{\it Winter USENIX Conference},
%Jan. 1994, pp. 355-372.

%\bibitem [Maju86] {Majumdar86}
%S. Majumdar, R. Bunt,
%``Measurement and Analysis of Locality Phases in File Referencing Behavior'',
%{\it SIGMETRICS},
%1986, pp. 180--192.

%\bibitem [Maxtor89]  {Maxtor89}
%Maxtor Corporation,
%``XT-8000S Product Specification and OEM Technical Manual'',
%Document 1015586, Revision B, July 1989.

%\bibitem [McElwee88] {McElwee88}
%M. McElwee,
%``The Real World of DASD Management'',
%{\it Computer Measurement Group (CMG) Conference},
%1988, pp. 672--677.

%\bibitem [McKusick84] {McKusick84}
%M. McKusick, W. Joy, S. Leffler, R. Fabry,
%``A Fast File System for UNIX'',
%{\it ACM Transactions on Computer Systems},
%Vol. 2, No. 3, August 1984, pp. 181--197.

%\bibitem [McKusick90] {McKusick90}
%M. McKusick, M. Karels, K. Bostic,
%``A pageable memory based filesystem'',
%{\it United Kingdom UNIX systems User Group (UKUUG) Summer Conference},
%%pub. United Kingdom UNIX systems User Group, Buntingford, Herts.,
%July 1990, pp. 9--13.

%\bibitem [McKusick94] {McKusick94}
%M. McKusick, T.J. Kowalski,
%``Fsck -- The UNIX File System Check Program'',
%{\it 4.4 BSD System Manager's Manual},
%O'Reilley \& Associates, Inc., Sebastopol, CA, 1994, pp. 3:1--21.

%\bibitem [McNutt86] {McNutt86}
%B. McNutt,
%``An Empirical Study of Variations in DASD Volume Activity'',
%{\it Computer Measurement Group (CMG) Conference},
%1986, pp. 274--283.

%\bibitem [McVoy91] {McVoy91}
%L. McVoy, S. Kleiman,
%``Extent-like Performance from a UNIX File System'',
%{\it Winter USENIX Conference},
%January 1991, pp. 1--11.

%\bibitem [Menon91] {Menon91}
%J. Menon, D. Mattson,
%``Performance of Disk Arrays in Transaction Processing Environments'',
%IBM Research Report RJ 8230,
%July 15, 1991.

%\bibitem [Menon92] {Menon92}
%J. Menon, J. Kasson,
%``Methods for Improved Update Performance of Disk Arrays'',
%{\it Hawaii International Conference on System Sciences},
%January 1992, \mbox{pp. 74--83}.

%\bibitem [Menon92a] {Menon92a}
%J. Menon, D. Mattson,
%``Performance of Disk Arrays in Transaction Processing Environments'',
%IBM Research Report RJ 8230,
%July 15, 1992.

%\bibitem [Menon93] {Menon93}
%J. Menon, J. Cortney,
%``The Architecture of a Fault-Tolerant Cached RAID Controller'',
%{\it IEEE International Symposium on Computer Architecture},
%May 1993, pp. 76--86.

%\bibitem [Mert70] {Mert70}
%A. G. Merten,
%``Some quantitative techniques for file organization'',
%Ph.D. Thesis,
%Technical Report No. 15,
%U. of Wisconsin Comput. Center,
%1970.

%\bibitem [Miller91] {Miller91}
%E. Miller, R. Katz,
%``Input/Output Behavior of Supercomputing Applications'',
%{\it Supercomputing},
%1991, pp. 567--576.

%\bibitem [Mitsuishi85] {Mitsuishi85}
%A. Mitsuishi, T. Mizoguchi, T. Miyachi,
%``Performance Evaluation for Buffer-Contained Disk Units'',
%{\it Systems and Computers in Japan},
%Vol. 16, No. 5, 1985, pp. 32--40.

%\bibitem [Miyachi86] {Miyachi86}
%T. Miyachi, A. Mitsuishi, T. Mizoguchi,
%``Performance Evaluation for Memory Subsystem of Hierarchical Disk-Cache'',
%{\it Systems and Computers in Japan},
%Vol. 17, No. 7, 1986, pp. 86--94.

%\bibitem [Mogul94] {Mogul94}
%J. Mogul,
%``A Better Update Policy'',
%{\it Summer USENIX Conference},
%1994, pp. 99--111.

%\bibitem [Moran87] {Moran87}
%J. Moran,
%``SunOS Virtual Memory Implementation'',
%{\it European UNIX Users Group (EUUG) Conference},
%Spring 1988, pp. 285--300.

%\bibitem [Mourad93] {Mourad93}
%A. Mourad, W.K. Fuchs, D. Saab,
%``Performance of Redundant Disk Array Organizations in Transaction Processing Environments'',
%{\it International Conference on Parallel Processing},
%Vol. I, 1993, pp. 138--145.

%\bibitem [Muchmore89] {Muchmore89}
%S. Muchmore,
%``A comparison of the EISA and MCA architectures'',
%{\it Electronic Engineering},
%March 1989, pp. 91--97.

%\bibitem [Mullender84] {Mullender84}
%S. Mullender, A. Tanenbaum,
%``Immediate Files'',
%{\it Software--Practice and Experience},
%14 (4), April 1984, pp. 365--368.

%\bibitem [Muller91] {Muller91}
%K. Muller, J. Pasquale,
%``A High Performance Multi-Structured File System'',
%{\it ACM Symposium on Operating Systems Principles},
%1991, pp. 56--67.

%\bibitem [Mummert95] {Mummert95}
%L. Mummert, M. Ebling, M. Satyanarayanan,
%``Exploiting Weak Connectivity for Mobile File Access'',
%Unpublished Report, Carnegie Mellon University,
%March 1995.

%\bibitem [Muntz90] {Muntz90}
%R. Muntz, J. Lui,
%``Performance Analysis of Disk Arrays Under Failure'',
%{\it International Conference on Very Large Data Bases},
%1990, pp. 162--173.

%\bibitem [Myers86] {Myers86}
%G. Myers, A. Yu, D. House,
%``Microprocessor Technology Trends'',
%{\it Proceedings of the IEEE},
%Vol. 74, December 1986, pp. 1605--1622.

\bibitem [NCR89] {NCR89}
NCR Corporation,
``NCR 53C700 SCSI I/O Processor Programmer's Guide'',
1989.

\bibitem [NCR90] {NCR90}
NCR Corporation,
``Using the 53C700 SCSI I/O Processor'',
SCSI Engineering Notes, No. 822, Rev. 2.5,
Part No. 609-3400634, February 1990.

%\bibitem [NCR90] {NCR90}
%NCR Corporation,
%``Understanding the Small Computer System Interface'',
%Prentice-Hall, Inc., Englewood Cliffs, New Jersey, 1990.

\bibitem [NCR91] {NCR91}
NCR Corporation,
``Class 3433 and 3434 Technical Reference'',
Document No. D2-0344-A, May 1991.

%\bibitem [Nelson88] {Nelson88}
%M. Nelson, B. Welch, J. Ousterhout,
%``Caching in the Sprite Network File System'',
%{\it ACM Transactions on Computer Systems},
%February 1988, pp. 134--154.

%\bibitem [Ng88] {Ng88}
%S. Ng, D. Lang, R. Selinger,
%``Trade-offs Between Devices and Paths In Achieving Disk Interleaving'',
%{\it IEEE International Symposium on Computer Architecture},
%1988, pp. 196--201. 

%\bibitem [Ng89] {Ng89}
%S. Ng,
%``Some Design Issues of Disk Arrays'',
%{\it COMPCON},
%Spring 1989, pp. 137--142.

%\bibitem [Ng91] {Ng91}
%S. Ng,
%``Improving Disk Performance Via Latency Reduction'',
%{\it IEEE Transactions on Computers},
%January 1991, pp. 22--30.

%\bibitem [Ng92] {Ng92}
%S. Ng, R. Mattson,
%``Maintaining Good Performance in Disk Arrays During Failure Via Uniform Parity Group Distribution'',
%{\it International Symposium on High-Performance Distributed Computing},
%September 1992, pp. 260--269.
%%IBM Research Report, May 29, 1992.

%\bibitem [Ng92a] {Ng92a}
%S. Ng,
%``Prefetch Policies For Striped Disk Arrays'',
%IBM Research Report RJ 9055,
%October 23, 1992.

%\bibitem [Ohta90] {Ohta90}
%M. Ohta, H. Tezuka,
%``A fast /tmp file system by delay mount option'',
%{\it Summer USENIX Conference},
%June 1990, pp. 145--150.

%\bibitem [Oney75] {Oney75}
%W. Oney,
%``Queueing Analysis of the Scan Policy for Moving-Head Disks'',
%{\it Journal of the ACM},
%Vol. 22, No. 3, July 1975, pp. 397--412.

%\bibitem [OpenMarket96] {OpenMarket96}
%``WebServer Technical Overview'',
%\mbox{Open Market, Inc., available as}
%{\it http://www.openmarket.com/library/WhitePapers/Server/index.html},
%March 3, 1996.

%\bibitem [Orji93] {Orji93}
%C. Orji, J. Solworth,
%``Doubly Distorted Mirrors'',
%{\it ACM SIGMOD International Conference on Management of Data},
%May 1993, pp. 307--316.

\bibitem [Otoole94] {Otoole94}
J. O'Toole, L. Shrira,
``Opportunistic Log: Efficient Installation Reads in a Reliable Storage Server'',
{\it USENIX Symposium on Operating Systems Design and Implementation (OSDI)},
November 1994, pp. 39--48.

%\bibitem [Ouchi78] {Ouchi78}
%N. Ouchi,
%``System for Recovering Data Stored in Failed Memory Unit'',
%U.S. Patent \#4,092,732,
%May 30, 1978.

\bibitem [Ousterhout85] {Ousterhout85}
J. Ousterhout, H. Da Costa, D. Harrison, J. Kunze, M. Kupfer, J. Thompson,
``A Trace-Driven Analysis of the UNIX 4.2 BSD File System'',
{\it ACM Symposium on Operating System Principles},
1985, pp. 15--24.

%\bibitem [Ousterhout89] {Ousterhout89}
%J. Ousterhout, F. Douglis,
%``Beating the I/O Bottleneck: A Case for Log-Structured File Systems'',
%{\it ACM Operating Systems Review},
%January 1989, pp. 11--28.

%\bibitem [Ousterhout90] {Ousterhout90}
%J. Ousterhout,
%``Why Aren't Operating Systems Getting Faster As Fast as Hardware?'',
%{\it Summer USENIX Conference},
%June 1990, pp. 247--256.

%\bibitem [Papy88] {Papy88}
%W. Papy,
%``DASD I/O Performance Tuning: A Case Study of Techniques and Results'',
%{\it Computer Measurement Group (CMG) Conference},
%1988, pp. 665--671.

%\bibitem [Patterson88] {Patterson88}
%D. Patterson, G. Gibson, R. Katz,
%``A Case for Redundant Arrays of Inexpensive Disks (RAID)'',
%{\it ACM SIGMOD International Conference on Management of Data},
%May 1988, pp. 109--116.

%\bibitem [Patterson93] {Patterson93}
%R.H. Patterson, G. Gibson, M. Satyanarayanan,
%``A Status Report on Research in Transparent Informed Prefetching'',
%{\it ACM Operating Systems Review},
%Vol. 27, No. 2, April 1993, pp. 21--34.

%\bibitem [Patterson95] {Patterson95}
%R.H. Patterson, et alia,
%``Informed Prefetching and Caching'',
%{\it Symposium on Operating Systems Principles},
%December 1995, pp. 79--95.

%\bibitem [Peacock88] {Peacock88}
%J.K. Peacock,
%``The Counterpoint Fast File System'',
%{\it USENIX Winter Conference},
%February 1988, pp. 243--249.

%\bibitem [Quantum96] {Quantum96}
%Quantum Corporation,
%http://www.quantum.com/products/atlas2/,
%June 1996.

%\bibitem [Ramakrishnan92] {Rama92}
%K. Ramakrishnan, P. Biswas, R. Karelda,
%``Analysis of File I/O Traces in Commercial Computing Environments'',
%{\it ACM SIGMETRICS International Conference on Measurement and Modeling of Computer Systems},
%1992, pp. 78--90.

%\bibitem [Rau79] {Rau79}
%B.R. Rau,
%``Program Behavior and the Performance of Interleaved Memories'',
%{\it IEEE Transactions on Computers},
%Vol. C-28, No. 3, March 1979, pp. 191--199.

%\bibitem [Reddy89] {Reddy89}
%A.L.N. Reddy, P. Banerjee,
%``An Evaluation of Multiple-Disk I/O Systems'',
%{\it IEEE Transactions on Computers},
%Vol. 38, No. 12, December 1989, \mbox{pp. 1680--1690}.

%\bibitem [Reddy90] {Reddy90}
%A.L.N. Reddy, P. Banerjee,
%``A Study of I/O Behavior of Perfect Benchmarks on a Multiprocessor'',
%{\it IEEE International Symposium on Computer Architecture},
%1190, pp. 312--321.

%\bibitam [Reddy91] {Reddy91}
%A.L.N. Reddy, P. Banerjee,
%``Gracefully Degradable Disk Arrays'',
%{\it Fault Tolerant Computing Symposium},
%1991, pp. 401--408.

%\bibitem [Reddy92] {Reddy92}
%A.L.N. Reddy,
%``Reads and Writes: When I/Os Aren't Quite the Same'',
%{\it Hawaii International Conference on System Sciences},
%January 1992, pp. XX--YY.

%\bibitem [Reddy92] {Reddy92}
%A.L.N. Reddy,
%``A Study of I/O System Organizations'',
%{\it IEEE International Symposium on Computer Architecture},
%May 1992, pp. 308--317.

%\bibitem [Reve75] {Revelle75}
%R. Revelle,
%``An Empirical Study of File Reference Patterns'',
%IBM Technical Report RJ 1557,
%April 21, 1975.

%\bibitem [Richardson92] {Richardson92}
%K. Richardson, M. Flynn,
%``TIME: Tools for Input/Output and Memory Evaluation'',
%{\it Hawaii International Conference on Systems Sciences},
%January 1992, pp. 58--66.

%\bibitem [Ritchie78] {Ritchie78}
%D. Ritchie, K. Thompson,
%``The UNIX Time-Sharing System'',
%{\it Bell System Technical Journal},
%Vol. 57, No. 6, July/August 1978, pp. 1905--1930.

%\bibitem [Ritchie86] {Ritchie86}
%D. Ritchie,
%``The UNIX I/O System'',
%UNIX User's Supplementory Document, University of California, Berkeley,
%April 1986.

%\bibitem [Rosenblum92] {Rosenblum92}
%M. Rosenblum, J. Ousterhout,
%``The Design and Implementation of a Log-Structured File System'',
%{\it ACM Transactions on Computer Systems},
%10 (1), February 1992, pp. 25-52.

%\bibitem [Rosenblum95] {Rosenblum95}
%M. Rosenblum, et alia,
%``The Impact of Architectural Trends on Operating System Performance'',
%{\it Symposium on Operating Systems Principles},
%Dec. 1995, pp. 285-298.

\bibitem [Rosenblum95] {Rosenblum95}
M. Rosenblum, S. Herrod, E. Witchel, A. Gupta,
``Complete Computer Simulation: The SimOS Approach'',
{\it IEEE Journal of Parallel and Distributed Technology},
Winter 1995, pp. 34-43.

%\bibitem [Ruemmler91] {Ruemmler91}
%C. Ruemmler, J. Wilkes,
%``Disk Shuffling'',
%Technical Report HPL-CSP-91-30, Hewlett-Packard Laboratories,
%October 3, 1991.

\bibitem [Ruemmler93] {Ruemmler93}
C. Ruemmler, J. Wilkes,
``UNIX Disk Access Patterns'',
{\it Winter USENIX Conference},
January 1993, pp. 405--420.

%\bibitem [Ruemmler93a] {Ruemmler93a}
%C. Ruemmler, J. Wilkes,
%``A Trace-Driven Analysis of Disk Working Set Sizes'',
%Technical Report HPL-OSR-93-23, Hewlett-Packard Laboratories,
%April 1993.

%\bibitem [Ruemmler93b] {Ruemmler93b}
%C. Ruemmler, J. Wilkes,
%``Modelling Disks'',
%Technical Report HPL-93-68, Hewlett-Packard Laboratories,
%July 1993.

\bibitem [Ruemmler94] {Ruemmler94}
C. Ruemmler, J. Wilkes,
``An Introduction to Disk Drive Modeling'',
{\it IEEE Computer},
Vol. 27, No. 3, March 1994, pp. 17--28.

%\bibitem [Salem86] {Salem86}
%K. Salem, G. Garcia-Molina,
%``Disk Striping'',
%{\it IEEE International Conference on Data Engineering},
%1986, pp. 336--342.

%\bibitem [Sandberg85] {Sandberg85}
%R. Sandberg, D. Goldberg, S. Kleiman, D. Walsh, B. Lyon,
%``Design and Implementation of the Sun Network Filesystem'',
%{\it Summer USENIX Conference},
%June 1985, pp. 119--130.

\bibitem [Satya86] {Satya86}
M. Satyanarayanan,
{\it Modeling Storage Systems},
UMI Research Press, Ann Arbor, MI, 1986.

\bibitem [Schindler99] {Schindler99}
J. Schindler, G. Ganger,
``Automated Disk Drive Characterization'',
Technical Report CMU-CS-99-176, Carnegie Mellon University,
December 1999.

%\bibitem [Schulze89] {Schulze89}
%M. Schulze, G. Gibson, R. Katz, D. Patterson,
%``How Reliable is a RAID?'',
%{\it COMPCON},
%Spring 1989, pp. 118--123.

%\bibitem [Scranton83] {Scranton83}
%R. A. Scranton, D. A. Thompson, D. W. Hunter,
%``The Access Time Myth'',
%IBM Research Report, RC 10197, September 21, 1983.

%\bibitem [SCSI93] {SCSI93}
%``Small Computer System Interface-2'',
%ANSI X3T9.2, Draft Revision 10k,
%March 17, 1993.

%\bibitem [Seagate90] {Seagate90}
%Seagate Technology, Inc.,
%``Elite Disc Drive ST41600N User's Manual (SCSI Interface)'',
%Seagate Technology, Inc., Publication Number 83327460-A, 1990.

%\bibitem [Seagate91] {Seagate91}
%Seagate Technology, Inc.,
%``World Class Data Storage'',
%Seagate Technology, Inc., Publication Number 1000-006, 1991.

%\bibitem [Seagate91a] {Seagate91a}
%Seagate Technology, Inc.,
%``SCSI Interface Specification, Small Computer System Interface (SCSI)
%Elite Product Family'',
%Seagate Technology, Inc., Document Number 64721702, Revision B, March 1991.

\bibitem [Seagate92] {Seagate92}
Seagate Technology, Inc.,
``SCSI Interface Specification, Small Computer System Interface (SCSI), Elite
Product Family'',
Document Number 64721702, Revision D, March 1992.

\bibitem [Seagate92a] {Seagate92a}
eagate Technology, Inc.,
``Seagate Product Specification, ST41600N and ST41601N Elite Disc Drive, SCSI
Interface'',
Document Number 64403103, Revision G,
%October 1992.

%\bibitem [Seagate96] {Seagate96}
%Seagate Technology, Inc.,
%http://www.seagate.com/stor/stortop.shtml,
%June 1996.

%\bibitem [Seaman66] {Seaman66}
%P. H. Seaman, R. A. Lind, T. L. Wilson
%``An analysis of auxiliary-storage activity'',
%{\it IBM System Journal},
%Vol. 5, No. 3, 1966, pp. 158--170.

%\bibitem [Seaman69] {Seaman69}
%P. Seaman, R. Soucy,
%``Simulating Operating Systems'',
%{\it IBM System Journal},
%Vol. 8, No. 4, 1969, pp. 264--279.

%\bibitem [Seltzer90] {Seltzer90}
%M. Seltzer, P. Chen, J. Ousterhout,
%``Disk Scheduling Revisited'',
%{\it Winter USENIX Conference},
%1990, pp. 313--324.

%\bibitem [Seltzer92] {Seltzer92}
%M. Seltzer,
%``File System Performance and Transaction Support'',
%Ph.D. Dissertation, University of California at Berkeley,
%1992.

%\bibitem [Seltzer93] {Seltzer93}
%M. Seltzer, K. Bostic, M. McKusick, C. Staelin,
%``An Implementation of a Log-Structured File System for UNIX'',
%{\it Winter USENIX Conference},
%January 1993, pp. 201--220.

%\bibitem [Seltzer95] {Selzter95}
%M. Seltzer, et alia,
%``File System Logging Verses Clustering: A Performance Comparison'',
%{\it USENIX Technical Conference},
%January 1995, pp. 249-264.

%\bibitem [Small94] {Small94}
%C. Small, M. Seltzer,
%``Vino: An Integrated Platform for Operating Systems and Database Research'',
%Technical Report TR-30-94, Harvard, 1994.

%\bibitem [Smith81] {Smith81}
%A. Smith,
%``Input/Output Optimization and Disks Architectures: A Survey'',
%{\it Performance and Evaluation},
%1981, pp. 104--117.

%\bibitem [Smith85] {Smith85}
%A. Smith,
%``Disk Cache -- Miss Ratio Analysis and Design Considerations'',
%{\it ACM Transactions on Computer Systems},
%Vol. 3, No. 3, August 1985, \mbox{pp. 161--203}.

%\bibitem [Smith94] {Smith94}
%K. Smith, M. Selzter,
%``File Layout and File System Performance'',
%Harvard University, Report TR-35-94, 1994.

%\bibitem [Smith96] {Smith96}
%K. Smith, M. Selzter,
%``A Comparison of FFS Disk Allocation Policies'',
%{\it USENIX Annual Technical Conference},
%January 1996, pp. 15-25.

%\bibitem [Solworth90] {Solworth90}
%J. Solworth, C. Orji,
%``Write-Only Disk Caches'',
%{\it ACM SIGMOD International Conference on Management of Data},
%May 1992, pp. 123--132.

%\bibitem [Solworth91] {Solworth91}
%J. Solworth, C. Orji,
%``Distorted mirrors'',
%{\it International Conference on Parallel and Distributed Information Systems},
%December 1991, pp. 10--17.

%\bibitem [Staelin91] {Staelin91}
%``Smart Filesystems'',
%{\it Winter USENIX Conference},
%1991, pp. 45--51.

%\bibitem [Stodolsky93] {Stodolsky93}
%D. Stodolsky, G. Gibson, M. Holland,
%``Parity Logging Overcoming the Small Write Problem in Redundant Disk Arrays'',
%{\it IEEE International Symposium on Computer Architecture},
%May 1993, pp. 64--75.

%\bibitem [Stonebraker81] {Stonebraker81}
%M. Stonebraker,
%``Operating System Support for Database Management'',
%{\it Communications of the ACM},
%24 (7), 1981, pp. 412--418.

%\bibitem [Stonebraker87] {Stonebraker87}
%M. Stonebraker,
%``The Design of the POSTGRES Storage System'',
%{\it Very Large DataBase Conference},
%September 1987, pp. 289--300.

%\bibitem [Tang95] {Tang95}
%D. Tang,
%``Benchmarking Filesystems'',
%Report TR-19-95, Harvard University,
%April 1995.

%\bibitem [Teorey72] {Teorey72}
%T. Teorey, T. Pinkerton,
%``A Comparative Analysis of Disk Scheduling Policies'',
%{\it Communications of the ACM},
%Vol. 15, No. 3, March 1972, \mbox{pp. 177--184}.

%\bibitem [Teradata85] {Teradata85}
%Teradata Corporation.
%``DBC/1012 Database Computer System Manual Release 2.0'',
%Document No. C10-0001-02,
%November, 1985.

%\bibitem [Thekkath92] {Thekkath92}
%C. Thekkath, J. Wilkes, E. Lazowska,
%``Techniques for File System Simulation'',
%Technical Report HPL-92-131, Hewlett-Packard Laboratories,
%October 1992.

\bibitem [Thekkath94] {Thekkath94}
C. Thekkath, J. Wilkes, E. Lazowska,
``Techniques for File System Simulation'',
{\it Software -- Practice and Experience},
Vol. 24, No. 11, November 1994, pp. 981--999.

%\bibitem [TPCB90] {TPCB90}
%Transaction Processing Performance Council,
%``TPC Benchmark B, Standard Specification'',
%Draft 4.1, August 23, 1990.

%\bibitem [Treiber94] {Treiber94}
%K. Treiber, J. Menon,
%``Simulation Study of Cached RAID 5 Designs'',
%IBM Research Report RJ 9823,
%May 23, 1994.

%\bibitem [Verk85] {Verkamo85}
%A.I. Verkamo,
%``Empirical Results on Locality in Database Referencing'',
%{\it SIGMETRICS},
%1985, pp. 49--58.

%\bibitem [Vongsathorn90] {Vongsathorn90}
%P. Vongsathorn, S. Carson,
%``A System for Adaptive Disk Rearrangement'',
%{\it Software -- Practice and Experience},
%Vol. 20, No. 3, March 1990, \mbox{pp. 225--242}.

%\bibitem [Wilhelm76] {Wilhelm76}
%N. Wilhelm,
%``An Anomoly in Disk Scheduling: A Comparison of FCFS and SSTF Seek
%Scheduling Using an Empirical Model for Disk Accesses'',
%{\it Communications of the ACM},
%Vol. 19, No. 1, January 1976, pp. 13--17.

%\bibitem [Wilkes95] {Wilkes95}
%``The HP AutoRAID hierarchical storage system'',
%{\it Symposium on Operating Systems Principles},
%December 1995, pp. 96-109.

%\bibitem [Wilmot89] {Wilmot89}
%R. Wilmot,
%``File Usage Patterns from SMF Data: Highly Skewed Usage'',
%{\it Computer Measurement Group},
%1989.

%\bibitem [Wolf89] {Wolf89}
%J. Wolf,
%``The Placement Optimization Program: A Practical Solution to the Disk
%File Assignment Problem'',
%{\it ACM SIGMETRICS International Conference on Measurement and Modeling of Computer Systems},
%May 1989, pp. 1--10.

%\bibitem [Worthington93] {Worthington93}
%B. Worthington, Y. Patt,
%``Spindle Synchronization in Disk Arrays: Asset or Liability?'',
%Unpublished report, University of Michigan,
%January 1993.

\bibitem [Worthington94] {Worthington94}
B. Worthington, G. Ganger, Y. Patt,
``Scheduling Algorithms for Modern Disk Drives'',
{\it ACM SIGMETRICS Conference},
May 1994, pp. 241--251.

%\bibitem [Worthington94a] {Worthington94a}
%B. Worthington, G. Ganger, Y. Patt,
%``Scheduling for Modern Disk Drives and Non-Random Workloads'',
%Report CSE-TR-194-94, University of Michigan, Ann Arbor,
%March 1994.

\bibitem [Worthington95] {Worthington95}
B. Worthington, G. Ganger, Y. Patt, J. Wilkes,
``On-Line Extraction of SCSI Disk Drive Parameters'',
{\it ACM SIGMETRICS Conference},
May 1995, \mbox{pp. 146--156}.

\bibitem [Worthington95a] {Worthington95a}
B. Worthington,
``Aggressive Centralized and Distributed Scheduling of Disk Requests'',
Ph.D. Dissertation, CSE-TR-244-95, University of Michigan, Ann Arbor,
June 1995.

\bibitem [Worthington96] {Worthington96}
B. Worthington, G. Ganger, Y. Patt, J. Wilkes,
``On-Line Extraction of SCSI Disk Drive Parameters'',
Technical Report, University of Michigan, Ann Arbor,
1996, in progress.

%\bibitem [Zhou87] {Zhou87}
%S. Zhou,
%``An Experimental Assessment of Resource Queue Lengths as Load Indices'',
%{\it Winter USENIX Conference},
%1987, pp. 73--82.

%\bibitem [Zhou88] {Zhou88}
%S. Zhou,
%``A Trace-Driven Simulation Study of Dynamic Load Balancing'',
%{\it IEEE Transactions on Software Engineering},
%Vol. 14, No. 9, September 1988, pp. 1327--1341.

\clearpage

\end{thebibliography}



\end{document}







